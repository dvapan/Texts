\pagebreak
\section*{Приложения}
\label{sec:appl}

\normalsize
\subsection*{Приложение 1}
\label{appl_src}

\begin{lstlisting}[language=Python,
caption={Интерфейс Impulse System}]
__author__ = 'dvapan'

import scipy as sc
from scipy.integrate import odeint
from scipy.linalg import norm
import logging
import matplotlib.pyplot as plt


delta = [(0, 1), (1, 0), (0, -1), (-1, 0)]


class ImpulseSystem(object):
    drift = None
    impulse = None

    def __init__(self, drift=None, impulse=None):
        self.drift = drift
        self.impulse = impulse
        self.x_0 = None
        self.ht = None
        self.h = None
        self.time = None
        self.value = None
        self.cost = dict()
        self.state_space = dict()
        self.logger = logging.getLogger('rs_counter.ImpulseSystem')
        self.logger.info('create ImpulseSystem object')
        self.grid_error = 0

    def integrate(self, x_0, ht, h, time, value):
        self.x_0 = x_0
        self.ht = ht
        self.h = h
        self.time = time
        self.value = value
        i, j = 0, 0
        points_queue = set([(i, j)])
        self.state_space[i, j] = 1.0 * value
        self.logger.info('start integrate system')
        if self.drift:
            for t in sc.arange(0, time, ht):
                self.logger.info('t = {0}'.format(t))
                self.impulse_jump(points_queue)
                self.make_drift()
                points_queue = set(self.state_space.keys())
        self.impulse_jump(points_queue)

    def impulse_jump(self, points_queue):
        self.logger.info('start making impulse jump')

        h = self.h

        counter = 1
        divider = (self.value * 1 / self.h) ** 2
        while points_queue:
            cur_point = points_queue.pop()
            i, j = cur_point
            if counter % divider == 0:
                self.logger.info('Working at {0} = {1} | iteration {2}'.format((i, j), self.state_space[i, j], counter))
            if not self.cost.get((i, j)):
                self.cost[i, j] = self.get_costs_at(i, j)
            for k in range(len(delta)):
                dx, dy = delta[k]
                c = self.cost[i, j][k]
                new_v = self.state_space[i, j] - c * h
                if new_v > 0 and self.state_space.get((i + dx, j + dy), 0) < new_v:
                    self.state_space[i + dx, j + dy] = new_v
                    points_queue.add((i + dx, j + dy))
            counter += 1

    def make_drift(self):
        self.logger.info('start making drift')
        next_lay = dict()
        dt = self.ht  # 1.0 * self.time / self.nt
        # logger.info("{0}".format(dt))
        h = self.h

        for (i, j) in self.state_space.keys():
            x, y = sc.array([i, j]) * h + self.x_0
            nx = sc.array(odeint(self.drift, (x, y), sc.linspace(0, dt, 10))[-1])
            #logger.info("{0} --> {1}".format((x, y), nx))
            new_i, new_j = map(int, (map(round, (nx - self.x_0) / h)))

            new_error = norm(sc.array([self.x_0[0] + i * h, self.x_0[1] + j * h]) - nx)

            next_lay[new_i, new_j] = self.state_space[i, j]
        self.state_space = next_lay

    def get_costs_at(self, i, j):
        h = self.h
        cur_matrix = self.impulse((self.x_0[0] + i * h, self.x_0[1] + j * h))
        return [self.get_cost(cur_matrix, 0, 1),
                self.get_cost(cur_matrix, 1, 0),
                self.get_cost(cur_matrix, 0, -1),
                self.get_cost(cur_matrix, -1, 0)]

    def get_cost(self, cur_matrix, x, y):
        try:
            v = sc.linalg.solve(cur_matrix, (x, y))
            if v[0] >= 0 and v[1] >= 0:
                return norm(v)
            else:
                return sc.NAN
        except:
            if cur_matrix[0, 0] == 0.0 and cur_matrix[1, 1] != 0.0:
                v = y / cur_matrix[1, 1]
                if v > 0:
                    return v
            if cur_matrix[1, 1] == 0.0 and cur_matrix[0, 0] != 0.0:
                v = x / cur_matrix[0, 0]
                if v > 0:
                    return v
            return sc.NAN


def get_part_hausdorphe(bp, points):
    return max(map(lambda p: min(map(lambda q: norm(p - q),
                                     points)),
                   sc.array(bp).transpose()))


def test_sample_without_drift(h,V):
    plt.clf()
    x_0 = [0, 0]
    F = None
    G = lambda x: sc.matrix([[1 - x[1], 0], [0, 1 - x[0]]])
    isys = ImpulseSystem(drift=F, impulse=G)
    isys.integrate(x_0, None, h, None, V)
    points = []
    bp = []
    for (i, j) in isys.state_space.keys():
        points.append([x_0[0] + i * h, x_0[1] + j * h])
    points = sc.array(points)
    sc.save('out/points_v_{0}_h_{1}'.format(V, h), points)
    logger.info('plotting approximated reachability set')
    if h <= 0.01 or V >= 5:
        plt.plot(points[:, 0], points[:, 1], ',')
    else:
        plt.plot(points[:, 0], points[:, 1], '.')
    logger.info('count Hausdorff distance')

    a = sc.linspace(0, 1, max(1.0/h, 100))
    plt.plot(sc.zeros(len(a)), a, 'r')
    bp.append(get_part_hausdorphe([sc.zeros(len(a)), a], points))
    logger.info('current distance {0}'.format(max(bp)))
    plt.plot(a, sc.zeros(len(a)), 'r')
    bp.append(get_part_hausdorphe([a, sc.zeros(len(a))], points))
    logger.info('current distance {0}'.format(max(bp)))

    a = sc.linspace(0, V, max(1.0/h, 100))
    r = 1 - sc.exp(-a / 2)
    y = r + (V - a) * (1 - r)
    plt.plot(r, y, 'r')
    bp.append(get_part_hausdorphe([r, y], points))
    logger.info('current distance {0}'.format(max(bp)))

    plt.plot(y, r, 'r')
    bp.append(get_part_hausdorphe([y, r], points))
    logger.info('current distance {0}'.format(max(bp)))
    if V > 1:
        if V <= 3:
            a = sc.linspace(V/2, V, max(1.0/h, 100))
        else:
            a = sc.linspace(V/4, V/2, max(1.0/h, 100))

        plt.plot(a, (1 - a) * (V - a), 'r')
        bp.append(get_part_hausdorphe([a, (1 - a) * (V - a)], points))
        logger.info('current distance {0}'.format(max(bp)))

        plt.plot((1 - a) * (V - a), a, 'r')
        bp.append(get_part_hausdorphe([(1 - a) * (V - a), a], points))
        logger.info('current distance {0}'.format(max(bp)))
    if V > 2:
        a = sc.linspace(0, V-2,  max(1.0/h, 100))
        r = sc.exp(a/2)
        plt.plot(1 - r, 1+r+(V-2-a)*r, 'r')
        bp.append(get_part_hausdorphe([1-r, 1+r+(V-2-a)*r], points))
        logger.info('current distance {0}'.format(max(bp)))
        plt.plot(1-r-(V-2-a)*r, 1 + r, 'r')
        bp.append(get_part_hausdorphe([1-r-(V-2-a)*r,1+r], points))
        logger.info('current distance {0}'.format(max(bp)))
        plt.plot(1+r+(V-2-a)*r, 1 - r, 'r')
        bp.append(get_part_hausdorphe([1+r+(V-2-a)*r, 1 - r], points))
        logger.info('current distance {0}'.format(max(bp)))
        plt.plot(1 + r, 1-r-(V-2-a)*r, 'r')
        bp.append(get_part_hausdorphe([1 + r, 1-r-(V-2-a)*r], points))
        logger.info('current distance {0}'.format(max(bp)))
    res = max(bp)
    f = open('out/data', 'a')
    f.write("{0} {1} {2}\n".format(V, h, res))
    f.close()
    plt.savefig("out/figure_v_{0}_h_{1}.png".format(V, h))
    #plt.show()


def test_sample_with_drift(h, ht):
    V = 1
    T = 13
    plt.clf()
    x_0 = [2, 0]
    G = lambda x: sc.matrix([[-x[0], 0], [0, -x[1]]])
    F = lambda x, t: [x[1]/sc.sqrt(x[1]**2+x[0]**2), -x[0]/sc.sqrt(x[1]**2+x[0]**2)]
    logger.info('count RS for h = {0}, ht = {1}'.format(h, ht))
    isys = ImpulseSystem(drift=F, impulse=G)
    isys.integrate(x_0, ht, h, T, V)
    points = []
    for (i, j) in isys.state_space.keys():
        points.append([x_0[0] + i * h, x_0[1] + j * h])
    points = sc.array(points)
    logger.info('count testable RS')
    isys2 = ImpulseSystem(drift=F, impulse=G)
    isys2.integrate(x_0, ht, h/2.0, T, V)
    points2 = []
    for (i, j) in isys2.state_space.keys():
        points2.append([x_0[0] + i * h, x_0[1] + j * h])
    points2 = sc.array(points2)
    
    logger.info('count Hausdorff distance')
    error = 0
    for p in points:
        min_h = min(map(lambda x: norm(x-p), points2))
        logger.info('min dist for point {0} = {1}. Current distance {2}'.format(p, min_h,error))
        if min_h>error:
            error = min_h
    
    sc.save('out/points_d_v_{0}_h_{1}'.format(V, h), points)
    logger.info('plotting approximated reachability set')
    if h < 0.05:
        plt.plot(points[:, 0], points[:, 1], ',')
    else:
        plt.plot(points[:, 0], points[:, 1], '.')
    # f = open('out/data', 'a')
    # f.write("{0} {1} {2}\n".format(ht, h, error))
    # f.close()
    plt.savefig("out/figure_d_h_{0}_ht_{1}.png".format(h, ht))


def test_sample_with_drift_counted(h,ht):
    V = 2
    T = 1
    plt.clf()
    x_0 = [1, 1]
    G = lambda x: sc.matrix([[x[1], 0], [0, x[0]]])
    #G = lambda x: sc.matrix([[0, 0], [0, 0]])
    F = lambda x, t: [1,1]
    logger.info('count RS for h = {0}, ht = {1}'.format(h, ht))
    isys = ImpulseSystem(drift=F, impulse=G)
    isys.integrate(x_0, ht, h, T, V)
    points = []
    bp = []
    for (i, j) in isys.state_space.keys():
        points.append([x_0[0] + i * h, x_0[1] + j * h])
    points = sc.array(points)
    logger.info('plotting approximated reachability set')
    if h < 0.05:
        plt.plot(points[:, 0], points[:, 1], ',')
    else:
        plt.plot(points[:, 0], points[:, 1], '.')
    f = open('log1.txt')
    s = f.read()
    lines = s.split('\r\n')[:-2]
    data = sc.array([map(float, l.split('    ')[1:]) for l in lines]).transpose()
    plt.plot(data[0,:-21], data[1,:-21],'r')
    plt.plot(data[0,21:], data[1,21:],'r')
    bp.append(get_part_hausdorphe([data[0,:], data[1,:]], points))
    x = sc.linspace(2,6,40)
    y = sc.full(len(x), 2)
    bp.append(get_part_hausdorphe([x,y], points))
    bp.append(get_part_hausdorphe([y,x], points))
    plt.plot(1,1,'.')
    plt.plot(x, y, 'r')
    plt.plot(y, x, 'r')
    res = max(bp)
    f = open('out/data', 'a')
    f.write("{0} {1} {2}\n".format(ht, h, res))
    f.close()
    sc.save('out/points_dc_ht_{0}_h_{1}'.format(ht, h), points)
    plt.savefig("out/figure_dc_h_{0}_ht_{1}.png".format(h, ht))


if __name__ == "__main__":
    logger = logging.getLogger('rs_counter')
    logger.setLevel(logging.DEBUG)
    # create file handler which logs even debug messages
    fh = logging.FileHandler('out/script.log')
    fh.setLevel(logging.DEBUG)
    # create console handler with a higher log level
    ch = logging.StreamHandler()
    ch.setLevel(logging.DEBUG)
    # create formatter and add it to the handlers
    formatter = logging.Formatter('%(asctime)s - %(name)s - %(levelname)s - %(message)s')
    fh.setFormatter(formatter)
    ch.setFormatter(formatter)
    # add the handlers to the logger
    logger.addHandler(fh)
    logger.addHandler(ch)

    seg_num = 100
    x_0 = [0, 0]

    ht=None
    h=0.01
    T=None
    V = 7
    F = None
    G = lambda x: sc.matrix([[-(1 - x[1])*(x[0]**2 + x[1]**2 - 2), 0], [0, -(1 - x[0])*(x[0]**2 + x[1]**2 - 2)]])
    isys = ImpulseSystem(drift=F, impulse=G)
    isys.integrate(x_0, ht, h, T, V)
    points = []
    bp = []
    for (i, j) in isys.state_space.keys():
        points.append([x_0[0] + i * h, x_0[1] + j * h])
    points = sc.array(points)
    logger.info('plotting approximated reachability set')
    if h < 0.05:
        plt.plot(points[:, 0], points[:, 1], ',')
    else:
        plt.plot(points[:, 0], points[:, 1], '.')
    plt.show()
    G = lambda x: sc.matrix([[-x[0], 0], [0, -x[1]]])
    F = lambda x, t: [x[1]/sc.sqrt(x[1]**2+x[0]**2), -x[0]/sc.sqrt(x[1]**2+x[0]**2)]
    f = open('out/data', 'w')
    f.write('ht,h,hausd_dist\n')
    f.close()
    V = 7
    test_sample_without_drift(V/10.0, V)

    for V in [1,2,5,7]:
        for h in [V/5.0, V/10.0, V/25.0, V/50.0, V/100.0]:
            logger.info('Start testing sample: V = {0}, h = {1}'.format(V, h))
            test_sample_without_drift(h, V)
    for ht in [1, 0.5, 0.2, 0.01]:
        for h in [0.2, 0.1, 0.05, 0.01]:
            logger.info('Start testing sample: ht = {0}, h = {1}'.format(ht, h))
            test_sample_with_drift_counted(h, ht)

    test_sample_with_drift(0.01, 0.1)

\end{lstlisting}
  



%%% Local Variables:
%%% mode: latex
%%% TeX-master: "paraniseikon"
%%% End:
