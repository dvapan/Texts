\pagebreak
\section*{Приложение А}
\addcontentsline{toc}{section}{Приложение А}
\label{sec:appl}

\vspace{1em}
Листинг А.1 --- Текст программы для решения уравнения эйконала FMM

\small
\begin{verbatim}

// -*- compile-command: "cc `pkg-config
 --libs --cflags glib-2.0 gsl` -lm solver.c " -*-

#include <stdio.h>
#include <math.h>
#include <glib.h>
#include <gsl/gsl_matrix.h>
#include <gsl/gsl_poly.h>

const double lbound_x = -8.0;
const double rbound_x = 8.0;
const double lbound_y = -8.0;
const double rbound_y = 8.0;

const double h      = 0.05;
const double x_step = 0.05;
const double y_step = 0.05;


gsl_matrix* T;
gsl_matrix* K;
GQueue* active_nodes;

typedef struct {
  int dim_1, dim_2;
  double *data;
} Grid;

typedef struct {
  int i,j;
} Node;

Node* make_node(int i, int j);
void node2point(Node node, 
      double* x, double *y);
void prt(gpointer item);
gint cmp(gconstpointer a, 
   gconstpointer b, gpointer _);

double g(Node node);
double speed(Node node);
gboolean is_internal(Node node);
/* get_neighbours(Node* node, GList nbours); */
void init();
void process_neighbour(int i, int j);


int main(void)
{
  int dim_1 = (rbound_x - lbound_x) / x_step;
  int dim_2 = (rbound_y - lbound_y) / y_step;
  T = gsl_matrix_alloc(dim_1, dim_2);
  K = gsl_matrix_calloc(dim_1, dim_2);

  /* printf("%d %d\n", dim_1, dim_2); */
  active_nodes = g_queue_new();

  init();

  while (!g_queue_is_empty(active_nodes)) {
    Node* node = g_queue_pop_head(active_nodes);
    gsl_matrix_set(K, node->i, node->j,0);
    process_neighbour(node->i - 1, node->j);
    process_neighbour(node->i + 1, node->j);
    process_neighbour(node->i, node->j - 1);
    process_neighbour(node->i, node->j + 1);
  }

  int i, j;
  for (i = 0; i < T->size1; ++i) {
    for (j = 0; j < T->size2; ++j){
      double val = gsl_matrix_get(T, i, j);
      printf("% lf % lf % lf\n", i*h, j*h, val);
    }
    putchar('\n');
  }
  return 0;
}

void process_neighbour(int i, int j)
{
  Node node = {i,j};
  if (!is_internal(node) || 
   !gsl_matrix_get(K, i, j)) return;
  double lT = gsl_matrix_get(T, i - 1, j);
  double rT = gsl_matrix_get(T, i + 1, j);
  double uT = gsl_matrix_get(T, i, j - 1);
  double dT = gsl_matrix_get(T, i, j + 1);

  double mini = lT < rT ? lT : rT;
  double minj = uT < dT ? uT : rT;

  double time;

  if (finite(mini) && isinf(minj))
    time = mini + x_step/speed(node);
  else if (finite(minj) && isinf(mini))
    time = minj + y_step/speed(node);
  else if (finite(mini) && finite(minj)){
    double a = 1/(x_step*x_step) + 
               1/(y_step*y_step);
    double b = -2 * (mini/pow(x_step, 2) +
                      minj/pow(y_step,2));
    double c = pow(mini/x_step, 2) + 
             pow(minj/y_step, 2) - 
               1.0/speed(node);
    double x1,x2;
    int croots = gsl_poly_solve_quadratic(a,b,c, &x1, &x2);
    if (croots == 0)
      time = INFINITY;
    else if (croots == 1)
      time = x1;
    else if (croots == 2)
      time = x2;
  }
  if (time < gsl_matrix_get(T, i, j)){
    gsl_matrix_set(T, i, j, time);
    g_queue_insert_sorted(active_nodes,
                          make_node(i,j),
                          (GCompareDataFunc)cmp,NULL);
  }
}

Node* make_node(int i, int j)
{
  Node* node = g_new(Node,1);
  node->i = i;
  node->j = j;
  return node;
}

void prt(gpointer item)
{
  int i = ((Node*)item)->i;
  int j = ((Node*)item)->j;
  printf("[%d,%d] = %g\n",i,j,gsl_matrix_get(T,i,j));
}

gint cmp(gconstpointer a, gconstpointer b, gpointer _)
{
  int i = ((Node*)a)->i;
  int j = ((Node*)a)->j;
  double val1 = gsl_matrix_get(T,i,j);
  i = ((Node*)b)->i;
  j = ((Node*)b)->j;
  double val2 = gsl_matrix_get(T,i,j);

  return floor(val1 - val2);
}

void node2point(Node node, double* x, double *y)
{
  *x = lbound_x + x_step * node.i;
  *y = lbound_y + y_step * node.j;
  return;
}

double g(Node node)
{
  double x, y;
  node2point(node, &x, &y);
  return x*x + y*y - 0.0001;
}

double speed(Node node)
{
  double x, y;
  node2point(node, &x, &y);
  double circle = pow(x,2) + pow(y - 3, 2) - 4;
  return 1 + 1.0/(pow(circle, 2) + 0.1);
}

gboolean is_internal(Node node) {
  int i = node.i;
  int j = node.j;
  return i > 0 && i < T->size1 - 1 && 
     j > 0 && j < T->size2 - 1;
}

void process_node(int i, int j)
{
  Node n1 = {.i = i, .j=j};
  if (g(n1) <= 0){
    gsl_matrix_set(T,i,j,0);
    gsl_matrix_set(K,i,j,1);
    if (is_internal(n1))
      g_queue_push_head(active_nodes,
                        make_node(i, j));
  } else {
    gsl_matrix_set(T,i,j,INFINITY);
    gsl_matrix_set(K,i,j,1);
  }

}
void init()
{
  int i, j;
  for (i = 0; i < T->size1; ++i)
    for(j = 0; j < T->size2; ++j)
      process_node(i, j);
}
\end{verbatim}
\large

\vspace{1em}
Листинг А.2 --- Текст программы для решение уравнения эйконала FSM
\small
\begin{verbatim}
// -*- compile-command: "cc -o sfs-fsm `pkg-config --libs
    --cflags glib-2.0 gsl` -lm sfs-fsm.c " -*-

#include <stdio.h>
#include <math.h>
#include <glib.h>
#include <unistd.h>
#include <stdbool.h>
#include <gsl/gsl_matrix.h>
#include <gsl/gsl_poly.h>

const int dim = 2;
double h = 1;

gsl_matrix* T;
gsl_matrix* F;

int sweep_dirs[2];

gboolean is_debug = 0;

void init();
double process_neighbour(int i, int j);

void get_sweep_dirs(/* int* sweep_dirs */);
bool sweep();

int main(int argc, char **argv)
{
  int dim_1,dim_2;
  int i, j;
  int opt;
  while ((opt = getopt(argc, argv, "d")) != -1) {
    switch (opt) {
    case 'd':
      is_debug = 1;
      puts("DEBUG:::OUTPUT\n");
      break;
    default: /* '?' */
      fprintf(stderr, "Usage: %s [-d]\n",
              argv[0]);
      exit(EXIT_FAILURE);
    }
  }

  /* puts("set grid\nset hidden3d\n" */
  /*      "$mapl << EOD"); */

  if (is_debug) puts("READ image size\n");
  scanf("%d%d", &dim_1, &dim_2);

  T = gsl_matrix_alloc(dim_1, dim_2);
  F = gsl_matrix_alloc(dim_1,dim_2);
  active_nodes = g_queue_new();
  init();

  bool stop = false;
  int sweep_count = 0;
  int pos[dim];
  for(int k = 1; k <=4; k++){
    get_sweep_dirs(sweep_dirs);
    /* stop = sweep(dim - 1,pos); */
    int _i, _j;
    for (int i = 0; i < F->size1; i++){
      for (int j = 0; j < F->size2; j++) {
        if (sweep_dirs[0] == 1)
          _i = i;
        else
          _i = F->size1 - i - 1;
        if (sweep_dirs[1] == 1)
          _j = j;
        else
          _j = F->size2 - j - 1;
        double time = process_neighbour(_i,_j);

        if (time < gsl_matrix_get(T, _i, _j)){
               
          gsl_matrix_set(T, _i, _j, time);
          /* stop = false; */
        } 

      }
    }
  }
  for (i = 0; i < T->size1; ++i) {
    for (j = 0; j < T->size2; ++j){
      double val = gsl_matrix_get(T, i, j);
      printf("%f %f %f\n", i*h,j*h,val);
    }
    putchar('\n');
  }

  /* puts("EOD\n" */
  /*      "splot '$mapl' matrix with lines notitle"); */
        
  return 0;
}

double process_neighbour(int i, int j)
{
  Node node = {i,j};
  /* if (!gsl_matrix_get(K, i, j)) return; */
  if (is_debug)
    printf("neighbour - [%d %d]\n", i, j );

  double lT = i > 0 ? 
    gsl_matrix_get(T, i - 1, j): 
   INFINITY;
  double rT = i < T->size1 - 1 ? 
   gsl_matrix_get(T, i + 1, j): 
   INFINITY;
  double uT = j > 0 ? 
   gsl_matrix_get(T, i, j - 1): 
   INFINITY;
  double dT = j < T->size2 - 1 ? 
   gsl_matrix_get(T, i, j + 1): 
   INFINITY;


  double mini = lT < rT ? lT : rT;
  double minj = uT < dT ? uT : dT;
  if (is_debug)
    printf("\t%g %g %g %g => %g %g\n", 
       lT, rT, uT, dT, mini, minj);

  double time = INFINITY;
  if (is_debug)
    printf("\tF[%d][%d] = %g\n", i,j,gsl_matrix_get(F, i, j));
  if (finite(mini) && isinf(minj)) {
    if (is_debug){
      printf("\t\tbranch->mini\n");
      printf("\t[%d %d]\n",i,j);
      printf("\t%g %g %g %g => %g %g\n",
          lT, rT, uT, dT, mini, minj);
    }
    time = mini + h/gsl_matrix_get(F, i, j);

  }
  else if (finite(minj) && isinf(mini)) {
    if (is_debug){
      printf("\t\tbranch->minj\n");
      printf("\t[%d %d]\n",i,j);
      printf(
       "\t%g %g %g %g => %g %g\n",
        lT, rT, uT, dT, mini, minj);
    }
    time = minj + h/gsl_matrix_get(F, i, j);


  }
  else if (finite(mini) && finite(minj)){
    if (is_debug)
      printf("\t\tbranch->both\n");
    double idx = 1.0/gsl_matrix_get(F, i, j);
    double a = 2;
    double b = -2 * (mini + minj); 
    double c = pow(mini, 2) + 
       pow(minj, 2) - 
       pow(h/gsl_matrix_get(F, i, j),2);
    double x1=NAN,x2=NAN;
    int croots = 
      gsl_poly_solve_quadratic(a,b,c, &x1, &x2);
    if (is_debug)
      printf(
       "\t%g %g %g = 0: %d roots: x1=%f x2=%f\n"
        ,a,b,c,croots, x1,x2 );
    if (croots == 0)
      time = INFINITY;
    else if (croots == 1)
      time = x1;
    else if (croots == 2)
      time = x2;
  }
  if (is_debug)
    printf("\ttime = %g \n",time );
  return time;
}

Node* make_node(int i, int j)
{
  Node* node = g_new(Node,1);
  node->i = i;
  node->j = j;
  return node;
}

void prt(gpointer item)
{
  int i = ((Node*)item)->i;
  int j = ((Node*)item)->j;
  printf("[%d,%d] = %g\n",
    i,j,gsl_matrix_get(T,i,j));
}

gint cmp(gconstpointer a, 
     gconstpointer b, gpointer _)
{
  int i = ((Node*)a)->i;
  int j = ((Node*)a)->j;
  double val1 = gsl_matrix_get(T,i,j);
  i = ((Node*)b)->i;
  j = ((Node*)b)->j;
  double val2 = gsl_matrix_get(T,i,j);

  return floor(val1 - val2);
}


gboolean is_internal(Node node) {
  int i = node.i;
  int j = node.j;
  return i > 0 && i < T->size1 - 1 
    && j > 0 && j < T->size2 - 1;
}


bool sweep(int n, int* pos)
{
  bool stop = true;
  if (n > 0){
    for (int i = 0; i < F->size2; i++){
      if (sweep_dirs[n] == 1)
        pos[n] = i;
      else
        pos[n] = F->size2 - i - 1;
      stop = sweep(n - 1, pos);
    }
  }
  else
    for (int i = 0; i < F->size1; i++){
      if (sweep_dirs[n] == 1)
        pos[n] = i;
      else
        pos[n] = F->size1 - i - 1;

      double time = 
       process_neighbour(pos[0],pos[1]);

      if (time < gsl_matrix_get(T, pos[0], pos[1])){
               
        gsl_matrix_set(T, pos[0], pos[1], time);
        stop = false;
      } 

    }
  return stop;

}

void get_sweep_dirs(/* int* sweep_dirs */)
{
  for(int i = 0; i < 2; i++) {
    sweep_dirs[i] += 2;
    if (sweep_dirs[i] <= 1)
      break;
    else
      sweep_dirs[i] = -1;
  }
}

void init()
{
  int i, j;
  for (i = 0; i < F->size1; ++i)
    for(j = 0; j < F->size2; ++j) {
      double buf;
      scanf("%lf",&buf);
      gsl_matrix_set(F,i,j,buf);
      gsl_matrix_set(T,i,j,INFINITY);
      /* gsl_matrix_set(K,i,j,1); */
    }

  sweep_dirs[0] = 1;
  sweep_dirs[1] = 1;
        
  long  points_count = 0, point_line = 0;
  scanf("%ld", &points_count);
  int start_i,start_j;

  for (point_line = 0; 
        point_line < points_count; 
          ++point_line) {
    scanf("%d %d",&start_i, &start_j);
    gsl_matrix_set(T,start_i,start_j,0);

  }

}
\end{verbatim}

%%% Local Variables:
%%% mode: latex
%%% TeX-master: "eikonal_solver"
%%% End:
