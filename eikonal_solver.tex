\documentclass[a4paper,12pt]{article}
\usepackage{indentfirst} % отделять первую строку раздела абзацным отступом тоже

\usepackage[english,russian]{babel}
\usepackage{mathptmx}
\usepackage{cyrtimes}

\usepackage{indentfirst}

%\usepackage[T2A]{fontenc}
\usepackage[utf8]{inputenc}
\usepackage[unicode=true]{hyperref}

\usepackage{amsmath,amssymb,amsfonts,longtable,hhline}
\usepackage{mathrsfs}
\usepackage{multimedia} 
\usepackage{clrscode}


\usepackage{listings}
\usepackage{color}

\usepackage{graphicx}
\usepackage{setspace}

\usepackage{setspace}


\usepackage{amssymb,amsfonts,amsmath,amsthm} % advanced math stuff
\usepackage{float} % pictures EXACTLY where it's placed [H]

%indent of first line of paragraph
\usepackage{indentfirst}
\parindent=1.25cm

% in-text styles
\usepackage{cancel} % cross-out text
\usepackage[normalem]{ulem} % advanced underlines
\usepackage{soulutf8}

\usepackage[usenames,dvipsnames,table]{xcolor} % names of colors
\usepackage{makecell}

\usepackage{tocloft}
\usepackage{import}
\usepackage{lastpage}
\usepackage{etoolbox}
\usepackage[title,titletoc]{appendix}
\usepackage{pdfpages} % this package allow you to include ready PDF-files
\usepackage{pdflscape} % landscape pages in pdf

\usepackage{hyperref}
\hypersetup{
  hidelinks
}


% some stuff for anvaced tables
\usepackage{tabularx}
\usepackage{multirow}
% also "multicol" allow you to write part of text in two columns
\usepackage{multicol}
\usepackage{booktabs}
\usepackage{array}
\setlength{\multicolsep}{-7pt}

% restrict hyphenations...
\hyphenpenalty=10000
% ...and restrict lines go out of \textwidth (smth like justify)
\sloppy

% date format
\usepackage{datetime}
\newdateformat{onlyyear}{\THEYEAR~г.}

% margin
\usepackage{geometry}
\geometry{left=3cm}
\geometry{right=1cm}
\geometry{top=1.5cm}
\geometry{bottom=2cm}
\geometry{heightrounded}
\geometry{marginparwidth=2.5cm}
%\geometry{marginparsep=2cm}

\usepackage{marginnote}
\renewcommand*{\marginfont}{\color{red}\footnotesize}

\reversemarginpar

\renewcommand{\marginnote}[2][]{}

\usepackage{afterpage}



\makeatletter
    \renewcommand{\l@section}{\@dottedtocline{1}{0.4cm}{0.4cm}}
    \renewcommand{\section}{\@startsection{section}{1}{0cm}{-3.5ex plus -1ex minus -.2ex}{2.3ex plus.2ex}{\raggedright\normalfont}}
\makeatother

\makeatletter
    \renewcommand{\l@subsection}{\@dottedtocline{1}{0.4cm}{0.4cm}}
    \renewcommand{\subsection}{\@startsection{subsection}{1}{0cm}{-3.5ex plus -1ex minus -.2ex}{2.3ex plus.2ex}{\raggedright\normalfont}}
\makeatother

\makeatletter
    \renewcommand{\l@subsubsection}{\@dottedtocline{1}{0.4cm}{0.4cm}}
    \renewcommand{\subsubsection}{\@startsection{subsubsection}{1}{0cm}{-3.5ex plus -1ex minus -.2ex}{2.3ex plus.2ex}{\raggedright\normalfont}}
\makeatother



%% style of unordered lists
\usepackage{enumitem}
\renewcommand{\labelitemi}{---}
\renewcommand{\labelenumi}{\asbuk{enumi})}
\renewcommand{\labelenumii}{\arabic{enumii})}
%% style of ordered lists
%% like 1.1.1 - third level
\renewcommand{\theenumi}{\arabic{enumi}}
\renewcommand{\labelenumi}{\arabic{enumi}.}
\renewcommand{\theenumii}{\arabic{enumii}}
\renewcommand{\labelenumii}{\arabic{enumi}.\arabic{enumii}.}
\renewcommand{\theenumiii}{\arabic{enumiii}}
\renewcommand{\labelenumiii}{\arabic{enumi}.\arabic{enumii}.\arabic{enumiii}.}



%Подписи к таблицам и рисункам
\usepackage[tableposition=top,font=large]{caption}
\usepackage{subcaption}
\DeclareCaptionLabelFormat{gostfigure}{Рисунок #2}
\DeclareCaptionLabelFormat{gosttable}{Таблица #2}
\DeclareCaptionLabelSeparator{gost}{~---~}
\captionsetup{labelsep=gost}
\captionsetup[figure]{labelformat=gostfigure}
\captionsetup[table]{labelformat=gosttable}
\renewcommand{\thesubfigure}{\asbuk{subfigure}}


% \renewcommand{\thefigure}{\thesection.\arabic{figure}}
% \renewcommand{\thetable}{\thesection.\arabic{figure}}
% \renewcommand{\theequation}{\thesection.\arabic{equation}}


% advanced headers'n'footers
\usepackage{fancyhdr}
\pagestyle{fancy}
\fancyhf{}
\fancyfoot[C]{\textcolor[gray]{0.4}{\thepage}}
\fancyheadoffset{0mm}
\fancyfootoffset{0mm}
\renewcommand{\headrulewidth}{0pt}
\renewcommand{\footrulewidth}{0pt}
\fancypagestyle{plain}{
	\fancyhf{}
	\cfoot{\textcolor[gray]{0.4}{\thepage}}
}


\usepackage{tocloft}
\renewcommand{\cfttoctitlefont}{\hspace{0.38\textwidth}\bfseries\large}
\renewcommand{\cftsecfont}{\normalsize{\hspace{1.25cm}}}
\renewcommand{\cftsubsecfont}{\hspace{1.25cm}}
\renewcommand{\cftbeforesecskip}{0em}
\setcounter{tocdepth}{4}
\makeatletter
\renewcommand{\l@section}{\@dottedtocline{1}{1.25cm}{0.5cm}}
\renewcommand{\l@subsection}{\@dottedtocline{1}{1.75cm}{1cm}}
\renewcommand{\l@subsubsection}{\@dottedtocline{1}{2.25cm}{1.25cm}}
\makeatother


\makeindex

\begin{document}

  \large
  % \begin{titlepage}
  \newpage
% \frenchspacing
\thispagestyle{empty} %нет нумерации 

{
  
  \begin{center}   
    Министерство образования и науки Российской Федерации\\ 
    \begin{spacing}{1.0} 
      федеральное государственное бюджетное образовательное учреждение \\ 
      высшего  образования \\ 
      «Иркутский государственный Университет» \\ 
      (ФГБОУ ВПО «ИГУ») 
    \end{spacing} 
    
    
    
    Институт математики, экономики и информатики 
    \\ 
    Кафедра методов оптимизации 
    
    
    
    ~\\[1.3cm] 
    
    { \bf 
      ВЫПУСКНАЯ КВАЛИФИКАЦИОННАЯ РАБОТА\\БАКАЛАВРА } 
    \\[0.3cm] 
    по направлению 01.03.02 Прикладная математика и информатика 
    \\ 
    профиль Математическое и компьютерное моделирование 
    \\[1.7cm] 
    
    { 
      Численная аппроксимация множества достижимости импульсной
      управляемой системы 
    } 
    
    
    ~\\[1.3cm] 
    
    \begin{multicols}{2} 
      \begin{flushleft} 
        \phantom{Четкая подпись} 
        \vspace{0.5cm} 
        
        \phantom{Четкая подпись} 
        \phantom{Четкая подпись} 
        \phantom{Четкая подпись} 
        \phantom{Четкая подпись} 
        \phantom{Четкая подпись} 
        \phantom{Четкая подпись} 
        \phantom{Четкая подпись} 
        \vspace{0.2cm} 
        
        \begin{spacing}{1.0} 
          Студента 4 курса очного отделения 
          \\группы 02422
          \\Апановича Данила Владимировича
          \vspace{0.5cm} 
          
          Руководитель: 
          \\д. ф.-м. н., профессор кафедры Методов оптимизации 
          \\ 
          \underline{\phantom{Четкая подпись}} Дыхта В.А. 
          \vspace{0.5cm} 
          
          Допущен к защите 
          \\Зав. кафедрой, д. ф.-м. н., профессор 
          \\ 
          \underline{\phantom{Четкая подпись}} Дыхта В.А. 
        \end{spacing} 
      \end{flushleft} 
    \end{multicols} 
    \vspace{1.3cm} 
    Иркутск~---~2016 г. 
  \end{center} 
}
\end{titlepage}

%%% Local Variables:
%%% mode: latex
%%% TeX-master: "rs-ids"
%%% End:

  %   \begin{codebox}
  %   \Procname{Сортировка вставками}
  %   \li \For $j \gets 2$ \To $\id{length}[A]$\\

  %   Hello there
  % \end{codebox}
  \setcounter{page}{4}

  \tableofcontents
  \pagebreak
  \chapter*{Введение}
\addcontentsline{toc}{chapter}{Введение}
\label{intro}

Здесь порассуждаем о МД и зачем оно нам нужно 


Оценка множества достижимости позволяет понять, как ведет себя система
при неизвестных внешних воздействиях.


Знание множества достижимости дает максимальную информацию о
возможностях системы и помогает находить оптимальные управления для
различных функционалов. Необходимость  исследования свойств пучков
разрывных траекторий и их МД связана с тем, что в механике полета,
электрофизике, медицине и экономике часто встречаются динамические
объекты, допускающие разрывные траектории, порожденные импульсными
управлениями.


Существует ряд прикладных задач, в которых возможны ситуации, когда за
очень короткий период времени происходит существенное изменение
некоторых параметров, описывающих функционирование системы.  Важные
примеры подобных ситуаций можно найти в математической экологии,
например, рациональное использование и охрана водных, земельных,
атмосферных, минеральных и энергетических ресурсов, аварийный разлив
нефти и моделирование распространения нефтяного пятна,  а также в
экономике, механике, ракетодинамике,  квантовой электронике (лазерное
излучение является импульсным по своей природе), робототехнике,
медиотерапии  и т.д. В частности, такие ситуации возникают при
моделировании реальных процессов, управление которыми осуществляется в
течение столь кратковременных промежутков, что их можно принимать как
мгновенные, а результаты воздействия приводят к быстрому изменению
процесса --- скачкам фазовой траектории моделируемой системы. Динамика
таких процессов в первом приближении может быть описана динамическими
системами  с разрывными траекториями и управлениями импульсного типа,
т.е. такими, при которых происходит резкое, скачкообразное изменение
состояния системы в отдельные моменты времени. 

  \pagebreak
  \section{Постановка задачи. Необходимая теория}
\label{sec:theory}

% Расскажем об импульсных динамических системах подробнее и подведем %
% соотв. теоретическую базу.
%\section {Импульсные управления}
\subsection {Уравнение эйконала}
\label{sec:csdisttrack}

Предположим - у нас есть некоторая граница (замкнутая кривая в
двумерном пространстве), разделяющая область $\Omega$ на две подобласти:
внутреннюю и внешнюю. Предположим также, что кривая распространяется с
известной скоростью $F$, как показано на рисунке~\ref{fig:eikvis}. Для наших
потребностей достаточно предположить, что кривая расширяется, т.е. ее
движение направлено вовне текущей области $(F>0)$, а также
игнорируется касательная компонента движения, т.е. кривая
распространяется только по нормальному вектору, определенному $F$:

\begin{figure}[h]
  \centering
  \includegraphics[width=0.5\linewidth]{img/eikonal_vision.png}
  \hfil \caption{распространение кривой со скоростью $F$}
  \label{fig:eikvis}

\end{figure}


Для того, чтобы определить положение кривой в каждой точке области. Мы
можем вычислить время прибытия $T$, когда впервые будет пересечена
каждая из точке $(x,y)$.

Если взять одномерный случай, то там мы можем вычислять расстояние как
произведение скорости на время, тогда мы можем записать уравнение для
функции $T$:

\begin{equation*}
  1 = F \frac{dT}{dx}
\end{equation*}

В пространствах более высокой размерности время прибытия T это решение
краевой задачи , также называемой уравнением эйконала.

\begin{equation}
  \label{eq:eikonal}
  \left\{ \begin{matrix}
      F(x) || \nabla T(x) || = 1, x \in \Omega \\
      T(x) = 0, x \in \Gamma
    \end{matrix}\right.
\end{equation}

Здесь $\Omega$ -- это область в $\mathbb{R}^n$, $\Gamma$ -- начальная
позиция кривой, $\nabla$ обозначает градиент, и $\|| \cdot ||$ является
Евклидовой нормой.

Стоит отметить, что эволюция кривой и время первого прибытия
отличаются. Для иллюстрации рассмотрим распространение кривой со
скоростью $F \equiv 1$, как представлено на Рисунке~\ref{fig:prpgt-eik}

\begin{figure}[h]
  \centering
  \includegraphics[width=0.3\linewidth]{img/propagate_eikonal.png}
  \hfil \caption{распространение кривой со скоростью $F = 1$}
  \label{fig:prpgt-eik}

\end{figure}

Жирная линия указывает где будет кривая на следующем шаге.  Предположим
теперь мы хотим заглянуть дальше, тогда мы получим следующую картину
на рисунке~\ref{fig:swallow-ex} кривая пройдет сквозь себя, соорудив
так называемый \textit{ласточкин хвост}. Чем он плох? Мы в некоторых
точках получаем многозначную функцию, чего нужно избегать. В
\cite{S1999}, Сетиан описывает эту ситуацию, как если мы рассмотрим
фронт распространения кривой, как фронт распространения огня, тогда
то, что было однажды сожжено, второй раз не сжигается.  Следовательно
нам стоит выбрать в каком-то смысле физически корректное решение с
фигурой, как на рисунке~\ref{fig:correct-exmp}

\begin{figure}[h]
  \centering
  \includegraphics[width=0.3\linewidth]{img/swallow-tail-example.png}
  \hfil \caption{Пример ласточкиного хвоста}
  \label{fig:swallow-ex}

\end{figure}

\begin{figure}[h]
  \centering
  \includegraphics[width=0.3\linewidth]{img/corrct-example.png}
  \hfil \caption{Пример корректного решения}
  \label{fig:correct-exmp}

\end{figure}

Уравнение~\eqref{eq:eikonal} является частным случаем \textit{уравнения
Гамильтона-Якоби} первого порядка, которое в статичном случае выглядит
следующим образом:

\begin{equation}
  \label{eq:hje}
  \left\{ \begin{matrix}
      H(x, Du) = 0,\text{на } \mathbb{R}^n \times (0,\infty) \\
      T(x) = 0,  \text{на } \mathbb{R}^n \times \{t = 0\},
    \end{matrix}\right.
\end{equation}
где Гамильтониан $H = H(x,Du)$ непрерывная вещественная функция на
$\mathbb{R}^n \times \mathbb{R}^n$ и $\phi : \mathbb{R}^n \rightarrow
\mathbb{R}$ -- начальная функция.

В общем случае это уравнение не имеет классических $C^1$
решений. Проблема эта имеет решение в обобщенных решениях, которые
непрерывны и удовлетворяют данному уравнению в частных производных
почти всюду.

%%% Local Variables:
%%% mode: latex
%%% TeX-master: "eikonal_solver"
%%% End:

  \pagebreak
  \section{Численные методы решения}
\label{sec:algrhtms}

Далее мы опишем алгоритмы использующиеся для решений уравнений
эйконала. Это два разных метода, базирующиеся на разных и идеях и,
следовательно, пригодные для разных целей. Один заточен для
максимального ускорения работы и, как это часто бывает с такими
алгоритмами практически неспособен к параллелизации, другой же --
наоборот работает несколько медленнее, но гораздо проще поддается
распараллеливанию.

\subsection{Общая идея методов}
\label{sec:general-idea}

Для всех методов нам необходимо научится решать задачу в локальном
случае. Разберем для начала одномерный случай. Пусть нам дано
следующее уравнение эйконала:

\begin{equation}
  \label{eq:eik_smp}
  \sqrt{(\frac{dT}{dx})^2}=F(x), T(-1) = T(1) = 0.
\end{equation}

% Вставить куда-нибудь ссылку
Для упрощения обозначений мы запишем производную $T$ по $x$ в виде
$T_x$. Также в будущем мы будем поступать и для частных
производных. После преобразования уравнение~\eqref{eq:eik_smp} примет вид:

\begin{equation*}
  \sqrt{T_x^2}=F(x), T(-1) = T(1) = 0.
\end{equation*}


Имеется функция скорости $F(x) > 0$, требуется построить
решение $T(x)$. Мы видим, что решение не уникально (если t(x) решает
задачу, то тогда и $-t(x)$ также ее решает). Будем работать только с
положительными решениями.

Рассмотрим обыкновенное дифференциальное уравнение и разобьем решение
на подзадачи:

\begin{equation*}
  \left\{
      \begin{array}{ll}
        T_x = ~F(x), \quad x \ge 0,\\
        T_x = -F(x), \quad x \le 0,\\[0.3cm]
        T(-1)= T(1) = 0.
      \end{array}
    \right.
\end{equation*}

Для численной аппроксимации разобьем ось $x$ на набор точек сетки
$x_i=i\Delta x$. Положим $T_i = T(i \Delta x)$ и $F_i = F(i \Delta
x)$, где $\Delta x$ - это шаг дискретизации, $i = -n, \cdots,
n$. Далее используем разложение Тейлора и отбросим остаток. Получим
следующую дискретную систему:

\begin{equation}
  \label{eq:discretise}
  \left\{
    \begin{array}{ll}
      T_n = 0,\\
      \frac{T_{i+1} - T_i}{\Delta x} = F_i, \quad i>0 \\\
      \frac{T_{i} - T_{i-1}}{\Delta x} = F_i, \quad i>0\\
      T(-n) = 0\\
    \end{array}
  \right.
\end{equation}

Отметим, что
\begin{itemize}
\item[ ] $T_{n-1}$ может быть получено из $T_n$
\item[ ] $T_{n-2}$ может быть получено из $T_{n-1}$
\item[ ]  $\cdots$
\item[ ] $T_{1}$ может быть получено из $T_2$
\item[ ] $T_{0}$ может быть получено из $T_{1}$
\item[ ]  $\cdots$
\item[ ] $T_{-n+1}$ может быть получено из $T_{-n}$
\item[ ] $T_{-n+2}$ может быть получено из $T_{-n+1}$
\item[ ]  $\cdots$
\item[ ] $T_{-1}$ может быть получено из $T_{-2}$
\item[ ] $T_{0}$ может быть получено из $T_{-1}$

\end{itemize}

Выше мы построили разностную схему, где вычисляем производные двигаясь
по направлению распространения границы: каждое следующее уравнения вне
текущей границы получено на основании уже имеющихся решений внутри
нее.

Вычисляя уравнение эйконала мы видим, что информация распространяется
как волны с определенной скоростью вдоль направления градиента. Наш
метод дискретизации вычисляет значения переменных используя
направления того откуда информация о решениях приходит. Если говорить
точнее, то дискретизация уравнений в частных производных использует
конечно-разностную трассировку смещающуюся в направлении знака
градиента.

Для одномерного случая у нас есть только два направления для каждой
точки $i$: правое $(i+1)$ и левое $(i - 1)$. Предположим, что у нас
есть решение в точке $i$, на итерации с номером $n$. Тогда мы имеем
два случая:
\begin{itemize}
\item Правое направление: $T_i^{n+1} = T_i^n - \Delta x D^{+x}_iT(x_i)$
\item Левое направление: $T_i^{n+1} = T_i^n - \Delta x D^{-x}_iT(x_i)$
\end{itemize}

Здесь мы приняли следующие обозначения из \eqref{eq:discretise}:

\begin{equation*}
  \begin{array}{ll}
    & D^{+x}_iT(x_i) = \frac{T_{i+1} - T_{i}}{\Delta x} \\
    & D^{-x}_iT(x_i) = \frac{T_{i} - T_{i-1}}{\Delta x}
  \end{array}
\end{equation*}

Подобным образом используя разложение Тейлора в $x1$ и $x2$ для
величины $T$ мы можем определить эти обозначения для двумерного
случая, задав тем самым регулярную двумерную сетку дискретизации,
тогда для точек $(x1,x2)$, с разбиением
$(x1_i,x2_i) = (i\Delta x,j \Delta y)$ и $T_{ij} = T(x1_i,x2_i)$ мы
получим:

\begin{equation}
  \label{eq:discrete-defines}
  \begin{array}{ll}
    D^{-x1}_{i,j}T(x1,x2) = \frac{T_{i,j} - T_{i-1,j}}{\Delta x1}  \\
    D^{+x1}_{i,j}T(x1,x2) = \frac{T_{i+1,j} - T_{i,j}}{\Delta x1}  \\
    D^{-x2}_{i,j}T(x1,x2) = \frac{T_{i,j} - T_{i,j-1}}{\Delta x2}  \\
    D^{+x2}_{i,j}T(x1,x2) = \frac{T_{i,j} - T_{i,j+1}}{\Delta x2}  \\
  \end{array}
\end{equation}

\begin{itemize}
\item $D^{+x1}$ вычисляет новое значение в узле сетки $(i,j)$ используя
  информацию из $i$ и $i+1$, таким образом информация для решения
  распространяется справа налево.

\item $D^{-x1}$ вычисляет новое значение в узле сетки $(i,j)$ используя
  информацию из $i$ и $i-1$, таким образом информация для решения
  распространяется слева направо.
\item $D^{+x2}$ вычисляет новое значение в узле сетки $(i,j)$ используя
  информацию из $j$ и $j+1$, таким образом информация для решения
  распространяется сверху вниз.

\item $D^{-x2}$ вычисляет новое значение в узле сетки $(i,j)$ используя
  информацию из $j$ и $j-1$, таким образом информация для решения
  распространяется снизу вверх.

\end{itemize}

Для двумерного случая разностные схемы действуют вдоль направления
градиента. На рисунке~\ref{fig:upwind-schema} иллюстрируются две
возможных ситуации. В первом случае распространение информации идет из
третьего квадранта, В другом случае -- из второго


\begin{figure}[h]
  \centering
  \includegraphics[width=\linewidth]{img/upwind-schema.png}
  \hfil \caption{Пример разных направлений распространения}
  \label{fig:upwind-schema}

\end{figure}

Крэндалл и Лайонс в \cite{V1983} доказали, что последовательные
монотонные схемы сходятся к корректному вязкостному решению.

\subsection{Fast sweeping method}
\label{sec:fast-sweeping-method}

Fast sweeping method (метод быстрых выметаний. обычно даже в
русских публикациях не переводится) далее FSM -- итеративный метод, который
использует разностную схему для дискретизации уравнения эйконала
\cite{F2005}. Идея, скрывающаяся за FSM это ``замести'' сетку в
определенных направлениях. Порядок выметания определяется
характеристиками соответствующего уравнения эйконала

\subsection{Fast marching method}
\label{sec:fast-marching-method}


\subsection{Нерегулярная сетка}
\label{sec:unstructured-mesh}

\subsubsection{Триангуляция}
\label{sec:triangulate}

\subsection{Вопросы реализации}
\label{sec:programming}



%%% Local Variables:
%%% mode: latex
%%% TeX-master: "eikonal_solver"
%%% End:

  \pagebreak
  \section{Примеры}
\label{sec:samples}

Разберем примеры работы алгоритма на двух примерах. Но для начала
введем несколько дополнительных сущностей. Первая из которых это как
на практике будет выглядеть проверка условия корректности Фробениуса,
о котором упоминалось в \eqref{sec:csdisttrack}. Для этого введем
понятие скобки Ли (коммутатора) двух гладких вектор-функций $\varphi(x)$,
$\psi(x)$ одной и той же размерности $n$, являющихся столбцами матрицы $G(x)$

Пусть $\varphi_x$, $\psi_x$ --- $n \times n$ матрицы частных
производных функций $\varphi$ и $\psi$, а $x$ --- вектор размерности $n$. В примерах рассматриваем
матрицу $G(x)$ размерности $n=2$. В этом случае функция $\varphi_x$ к
примеру будет выглядеть так:
\begin{equation*}
  \begin{pmatrix}
    \frac{d \varphi_1}{d x_1} & \frac{d \varphi_1}{d x_2} \\
    \frac{d \varphi_2}{d x_1} & \frac{d \varphi_2}{d x_2}
  \end{pmatrix}(x)
\end{equation*}
Матрица $\psi_x$ --- выглядит аналогично. Тогда скобка Ли
$[\varphi,\psi]$ функций $\varphi$, $\psi$ определяется равенством.
\begin{equation*}
  [\varphi, \psi](x) = \varphi_x(x) \psi(x) - \psi_x(x) \varphi(x)
\end{equation*}
Очевидно, что она антикоммутативна т.е. $[\varphi, \psi] = - [\psi,
\varphi]$.

При условии того, что мы используем матрицу $2 \times 2$, мы будем
говорить, что рассматриваемая управляемая система удовлетворяет
условию корректности Фробениуса, если выполняется равенство

\begin{equation}
  \label{eq:1}
  [\varphi,\psi](x) \equiv 0.
\end{equation}

Второе о чем стоит поговорить это способ, благодаря которому мы
оцениваем порядок точности полученного результата --- между двумя множествами
вычисляется расстояние Хаусдорфа \eqref{eq:hausd_dist}.

\subsection{Пример системы без дрейфа}
\label{sec:snwd}

Рассмотрим систему 

\begin{equation*}
  \begin{aligned}[b]
    &\dot{x_1}(t) = (1 -x_2(t))v_1(t), & x_1(0)=0\\
    &\dot{x_2}(t) = (1-x_1(t))v_2(t), & x_2(0) = 0\\[8pt]
    &v_1(t) \ge 0, v_2(t) \ge 0 \\
    &\int_{0}^{1} |v_1(t) + v_2(t)| dt \le V
  \end{aligned}
\end{equation*}


Стоит отметить, что для этой системы не выполнено условие корректности
Фробениуса (равенства нулю коммутатора). Покажем это. Возмем скобку
Ли (коммутатор) от двух гладких вектор-функций $\varphi$ и $\psi$,
являющимися столбцами матрицы

\begin{equation*}
  \begin{aligned}[b]
    G(x) = 
    \begin{pmatrix}
      \varphi & \psi
    \end{pmatrix},
    &
    \varphi =
    \begin{pmatrix}
      \varphi_1\\ \varphi_2
    \end{pmatrix},
    &
    \psi =
    \begin{pmatrix}
      \psi_1 \\ \psi_2
    \end{pmatrix}
  \end{aligned}
\end{equation*}


Скобка Ли тогда примет иметь вид ---

\begin{equation*}
  [\varphi,\psi] = \varphi_x \psi - \psi_x \varphi = 
  \begin{pmatrix}
    \frac{d \varphi_1}{d x_1} & \frac{d \varphi_1}{d x_2} \\
    \frac{d \varphi_2}{d x_1} & \frac{d \varphi_2}{d x_2}
  \end{pmatrix}
  \begin{pmatrix}
    \psi_1 \\ \psi_2
  \end{pmatrix}
  -
  \begin{pmatrix}
    \frac{d \psi_1}{d x_1} & \frac{d \psi_1}{d x_2} \\
    \frac{d \psi_2}{d x_1} & \frac{d \psi_2}{d x_2}
  \end{pmatrix}
  \begin{pmatrix}
    \varphi_1 \\ \varphi_2
  \end{pmatrix}
\end{equation*}

В этом примере матрица $G(x)$ имеет вид 

\begin{equation*}
  G(x) = 
  \begin{pmatrix}
    (1-x_2) & 0 \\
    0 & (1-x_1)
  \end{pmatrix}
\end{equation*}
В результмате несложных расчетов можно убедится, что условие корректности здесь не выполняется:
\begin{equation*}
  [\varphi,\psi] = 
  \begin{pmatrix}
    x_1 - 1\\
    x_2 - 1
  \end{pmatrix}
  \neq 0
\end{equation*}


Для этой системы была аналитически вычислена граница множества
достижимости в работе на основе работы \cite{AVS2016} Для разных
значений $V$ граница опиcывается немного по разному, но общий вид
формул отсается неизменным. На графиках красным цветом обозначены
линии, являющиеся границей.

%TODO: Нужны ли формулы?
На Рис.~\ref{fig:v11}-Рис.~\ref{fig:v72} представлены множества достижимости при различных
значениях ресурса $V$ и разных шагах сетки
\begin{figure}[h]
\centering
\begin{minipage}[h]{0.47\linewidth}
  \noindent \hfil
  \includegraphics[width=1\linewidth]{img/figure_v_1_h_01.png}
  \hfil \caption{Множество достижимости для системы с V=1, h=0.2}
  \label{fig:v11}
\end{minipage}
\hfill
\begin{minipage}[h]{0.47\linewidth}
  \noindent \hfil
  \includegraphics[width=1\linewidth]{img/figure_v_1_h_002.png}
  \hfil \caption{Множество достижимости для системы с V=1, h=0.02}
  \label{fig:v12}
\end{minipage}
\end{figure}
%\vfill
\begin{figure}[h]
\begin{minipage}[h]{0.47\linewidth}
  \noindent \hfil
  \includegraphics[width=1\linewidth]{img/figure_v_2_h_02.png}
  \hfil \caption{Множество достижимости для системы с V=2, h=0.2}
  \label{fig:v21}
\end{minipage}
\hfill
\begin{minipage}[h]{0.47\linewidth}
  \noindent \hfil
  \includegraphics[width=1\linewidth]{img/figure_v_2_h_002.png}
  \hfil \caption{Множество достижимости для системы с V=2, h=0.02}
  \label{fig:v22}
\end{minipage}
\vfill
\begin{minipage}[h]{0.47\linewidth}
  \noindent \hfil
  \includegraphics[width=1\linewidth]{img/figure_v_7_h_07.png}
  \hfil \caption{Множество достижимости для системы с V=7, h=0.7}
  \label{fig:v71}
\end{minipage}
\hfill
\begin{minipage}[h]{0.47\linewidth}
  \noindent \hfil
  \includegraphics[width=1\linewidth]{img/figure_v_7_h_014.png}
  \hfil \caption{Множество достижимости для системы с V=7, h=0.14}
  \label{fig:v72}
\end{minipage}
\end{figure}

Для различных значений ресурса $V$ и шага сетки $h$ мы вычислили
расстояние хаусдорфа между теоретичесской оценкой множества
достижимости и тем, что получилось в результате работы алгоритма

\begin{table}[f]
  \centering
  \begin{tabular}{|*{3}{c|}}
    $V$&$h$&$\rho(A,B)$\\ \hline
    1&0.2&0.19357953184\\
    1&0.1&0.0989472244814\\
    1&0.04&0.04\\
    1&0.02&0.02\\
    1&0.01&0.01\\
    2&0.4&0.335023013171\\
    2&0.2&0.193935429287\\
    2&0.08&0.080131031276\\
    2&0.04&0.04\\
    2&0.02&0.02\\
    3&0.6&0.6\\
    3&0.3&0.29644979071\\
    3&0.12&0.130019551125\\
    3&0.06&0.0659015266245\\
    3&0.03&0.0309661912566\\
    5&1.0&1.14082938538\\
    5&0.5&0.5\\
    5&0.2&0.205052066625\\
    5&0.1&0.0985691231869\\
    5&0.05&0.0470656986497\\
    7&1.4&1.89238926658\\
    7&0.7&1.0191890624\\
    7&0.28&0.363069883153\\
    7&0.14&0.139983207703\\
    7&0.07&0.0724479673101\\
  \end{tabular}
  \caption{Расстояние Хаусдорфа в зависимости от шага сетки для
    различных значений ресурса}
  \label{tab:hsd_ndft}
\end{table}

\pagebreak
\subsection{Пример системы с дрейфом}
\label{sec:swd}

Интегральные кривые этого примера при нулевом управлении
представляют собой окружности с центром в начале координат, причем
фазовая скорость по модулю всюду равена 1. Как и в предыдущем примере,
управления неотрицательны.

\begin{equation*}
  \begin{aligned}[b]
    &\dot{x_1}(t) = \frac{x_2}{\left\|x\right\|} + x_1v_1, &x_1(0) =
    0,\quad v_1(t) \ge 0;\\
    &\dot{x_2}(t) = \frac{-x_1}{\left\|x\right\|} + x_2v_2, &x_2(0)
    = 0,\quad v_2(t) \ge 0.\\
    &\int\limits_0^{13} |v_1(t)| + |v_2(t)| dt \le 1
  \end{aligned}
\end{equation*}
Здесь $\left\| \cdot \right\|$ --- евклидова норма.

Для этого примера условие корректности
Фробениуса выполняется. 
В этом примере матрица $G(x)$ имеет вид 

\begin{equation*}
  G(x) = 
  \begin{pmatrix}
    x_1 & 0 \\
    0 & x_2
  \end{pmatrix}
\end{equation*}
И нетрудно убедиться, что здесь   $[\varphi,\psi] \equiv 0$


\begin{figure}[h]
\centering
  \noindent \hfil
  \includegraphics[width=1\linewidth]{img/figure_d_h_001_ht_01}
  \hfil \caption{Множество достижимости для системы с V=1, h=0.02}
  \label{fig:v1h0.02}
\end{figure}


%%% Local Variables:
%%% mode: latex
%%% TeX-master: "rs-ids"
%%% End:

  \pagebreak
  % \section{Программа}
\label{sec:program}
краткая информация о реализации на Python. Полный исходник в
приложении.
%%% Local Variables:
%%% mode: latex
%%% TeX-master: "rs-ids"
%%% End:

  \section*{Заключение}
\label{sec:conclusion}

В дипломной работе были рассмотрены два алгоритма аппроксимации
множеств достижимости для импульсных управляемых систем с билинейной
структурой. Алгоритмы реализованы в среде Scientific Python; выполнен
ряд численных экспериментов; проведено сравнение с аналитическими
оценками множества достижимости в тех случаях, когда это возможно;
найдена погрешность алгоритма (расстояние Хаусдорфа между множеством
достижимости и его аппроксимацией). На основе полученных данных
сделаны следующие выводы:

1) Переборный алгоритм может быть применен лишь для грубой
аппроксимации в наиболее простых случаях (размерность 1-2), поскольку
объем вычислений экспоненциально зависит от числа участков, на которые
разбит временной интервал.

2) Даже в относительно простых примерах время работы пиксельного
алгоритма на 1-2 порядка меньше, чем у переборного.

3) Пиксельный метод для систем с билинейной структурой весьма
эффективен: погрешность алгоритма во всех рассмотренных случаях близка
к шагу разбиения фазового пространства.

Следует отметить, что в рамках дипломной работы данные алгоритмы
применялись к довольно узкому классу задач. Некоторые расширения этого
класса (отсутствие ограничения на знак управлений) не требуют
существенной переработки алгоритмов и программ. Изменение нормы, в
которой задано ограничение на векторное управление, приведет к
заметному усложнению алгоритма.

Требует дальнейшего исследования вопрос об эффективном выборе
дискретизации фазового пространства: целесообразно измельчать сетку в
окрестности множества, на котором функция цены не дифференцируема.

%%% Local Variables:
%%% mode: latex
%%% TeX-master: "rs-ids"
%%% End:

  \pagebreak
  \pagebreak
\begin{thebibliography}{99}
  \addcontentsline{toc}{chapter}{Список литературы}

\bibitem{DS2011} { Dykhta~V.A., Samsonyuk~O.N.}  Some applications of
  Hamilton-Jacobi inequalities for classical and impulsive optimal
  control problems // European Journal of Control. Nonsmooth analysis,
  Control and Optimization. 2011. Vol.17. Pp. 55--69.

\bibitem{ZS1991} { Завалищин С.Т., Сесекин А.Н.}  {Импульсные
    процессы: модели и приложения}.  М.: Наука, 1991.

\bibitem{MR2005} { Миллер Б.М., Рубинович Е.Я.}  { Оптимизация
    динамических систем с импульсными управлениями}.  М.: Наука, 2005.

\bibitem{DS2003} { Дыхта В.А., Самсонюк О.Н.}  Оптимальное импульсное
  управление с приложениями.  М.: Физматлит, 2003.

\bibitem{SS2010} { Самсонюк~О.Н., Сесекин~А.Н.} Оценки и свойства
  интегральных воронок траекторий нелинейных импульсных систем //
  Тез. докл. II Международной школы-семинара «Нелинейный анализ и
  экстремальные задачи». Иркутск, ИДСТУ СО РАН, 28 июня -- 4 июля 2010
  г. 2010. С. 64--65.

\bibitem{BS2005} Вдовина О.И., Сесекин А.Н. Численное построение
  областей достижимости для систем с импульсными управлениями //
  Тр. Ин-та математики и механики УрО РАН.  2005. Т. 11, № 1.

\bibitem{G1997} Гурман В.И. Принцип расширения в задачах
  управления. 2-е изд., перераб. и доп. М.:Физматлит, 1997, 288 с.

\bibitem{M1993} Миллер Б.М. Метод разрывной замены времени в задачах
  оптимального управления импульсными и дискретно-непрерывными
  системами // Автоматика и телемеханика. 1993. № 12. С. 3--32.

\bibitem{MR1995} Motta M., Rampazzo F. Space-time trajectories of
  nonlinear systems driven by ordinary and impulsive controls //
  Differential Integral Equations. 1995. Vol. 8.  Pp. 269-288.

\bibitem{MS1999} Motta M., Sartori C. Discontinuous solutions to
  unbounded differential inclusions under state constraints.
  Applications to optimal control problems // Set-Valued
  Analysis. 1999. Vol. 7. Pp. 295-322.

\bibitem{WZ2007} Wolenski P.R., Zabic S.  A Sampling Method and
  Approximation Results for Impulsive Systems // SIAM J. Control
  Optim., 2007, Vol. 46. Pp. 983-998.

\bibitem{AVS2016} {Д. В. Апанович, В. А. Воронов, О. Н. Самсонюк},
  Построение множества достижимости двумерной импульсной управляемой
  системы с билинейной структурой// Изв. Иркутского
  гос. ун-та. Сер. Математика, 15 (2016), 3–16

\bibitem{DS2000} {Дыхта В. А., Самсонюк О.Н.} Оптимальное импульсное
  управление с приложениями. — М.: ФИЗМАТ ЛИТ, 2000. — 256 с. — ISBN
  5-9221-0097-1.

\bibitem{S1999} {J.A. Sethian.} Level Set Methods and Fast Marching Methods: evolving
interfaces in computational geometry, computer vision and matherial
science. — 2nd Ed. — Cambridge Univ. Press, 1999

\end{thebibliography}


%%% Local Variables:
%%% mode: latex
%%% TeX-master: "rs-ids"
%%% End:

  \section*{Приложения}
\label{sec:appl}

Исходники 


%%% Local Variables:
%%% mode: latex
%%% TeX-master: "rs-ids"
%%% End:


  

  
\end{document}

%%% Local Variables:
%%% mode: latex
%%% TeX-master: "eikonal_solver"
%%% End:

\grid
