\chapter{Примеры}
\label{ch:samples}

Разберем примеры работы алгоритма на двух примерах.
\section{Пример системы без дрейфа}
\label{sec:sample}

Рассмотрим систему 

\begin{equation*}
  \begin{aligned}[b]
    &\dot{x_1}(t) = (1 -x_2(t))v_1(t), & x_1(0)=0\\
    &\dot{x_2}(t) = (1-x_1(t))v_2(t), & x_2(0) = 0\\[8pt]
    &v_1(t) \ge 0, v_2(t) \ge 0 \\
    &\int_{0}^{1} |v_1(t) + v_2(t)| dt \le V
  \end{aligned}
\end{equation*}

Для этой системы была аналитически вычислена граница множества достижимости в
работе на основе работы \cite{AVS2016}

\section{Пример системы с дрейфом}
\label{sec:sample}

Интегральные кривые следующей системы при нулевом управлении
представляют собой окружности с центром в начале координат, причем
модуль фазовой скорости всюду равен 1. Как и в предыдущем примере,
управления неотрицательны

\begin{equation*}
  \begin{aligned}[b]
    &\dot{x_1}(t) = \frac{x_2}{\left\|x\right\|} + x_1v_1, &x_1(0) =
    0,\quad v_1(t) \ge 0;\\
    &\dot{x_2}(t) = \frac{-x_1}{\left\|x\right\|} + x_2v_2, &x_2(0)
    = 0,\quad v_2(t) \ge 0.\\
    &\int\limits_0^{13} |v_1(t)+ v_2(t)| dt \le 1
  \end{aligned}
\end{equation*}
Здесь $\left\| \cdot \right\|$ --- евклидова норма.


%%% Local Variables:
%%% mode: latex
%%% TeX-master: "rs-ids"
%%% End:
