\section{Примеры}
\label{sec:samples}

\subsection{Простейшее уравнение эйконала}
\label{sec:simple-eikonal}

Рассмотрим несколько примеров и приложений для алгоритмов, описанных
выше. Для начала рассмотрим простой пример:


\begin{equation}
  \label{eq:eik-sur}
  \left\{ \begin{matrix}
      \| \nabla T(x) \| = 1, x \in \Omega \\
      T(x) = 0, x \in \Gamma
    \end{matrix}\right.
\end{equation}

В простейшем случае, мы считаем, что функция $F$ уравнение
\eqref{eq:eikonal} является константой и равна единице, т.е. после
дискретизации в каждой точке $\Omega$ функция $F$ принимает значение
$1$. Тогда решением будет концентрические окружности
рисунок~\ref{fig:eikonal-surface}, если же его рисовать в трехмерном
пространстве (см. Рисунок~\ref{fig:eikonal-surface-3d}).

\begin{figure}[H]
  \centering
  \includegraphics[width=\linewidth]{img/eikonal_simple_surface.png}
  \hfil \caption{Линии уровня численного решения эйконала с $F\equiv 1$ }
  \label{fig:eikonal-surface}
\end{figure}

\begin{figure}[H]
  \centering
  \includegraphics[width=\linewidth]{img/eikonal_simple_3d.png}
  \hfil \caption{Трехмерный график численного решения эйконала с $F\equiv 1$}
  \label{fig:eikonal-surface-3d}
\end{figure}

Если мы добавим барьер для распространения волн, т.е. в точках лежащих
на некотором прямоугольнике установим значение $0$. Мы получим
следующий результат. Смотри рисунок~\ref{fig:barier_surface}.

\begin{figure}[H]
  \centering
  \includegraphics[width=\linewidth]{img/barier_surface.png}
  \hfil \caption{Линии уровня численного решения эйконала с $F\equiv
    1$, с препятствием}
  \label{fig:barier_surface}
\end{figure}

\subsection{Множество достижимости импульсной системы}
\label{sec:ids}

В работах \cite{AVS2016, AV2015_1,AV2015_2} предлагалось считать
множество достижимости импульсной управляемой системы. В
использованном в тех работах методе были существенные ограничения на
систему ограничений на управление, а именно в том, что для описаний
ограничений на управлении использовалась $l_1$ метрика. Методы
численных решений уравнений эйконала позволяют нам использовать
евклидову метрику. 

Для начала взглянем на обычную
управляемую систему, расширение множества траекторий которой и
приводит к импульсной системе. Будем работать с системой следующего
вида.

\begin{equation}
  \label{system_s}
  \begin{array}{l}
    \dot{x}(t)=\big(x(t)\big)+G\big(x(t)\big)v(t), \quad x(0)=x_0, \\[8pt]
    v(t)\geq 0  \qquad \forall\, t\in T = [0,t_1], \qquad
    \displaystyle\int_{0}^{t_1} ||v(t)||dt\leq V.
  \end{array} 
\end{equation}

Здесь
\begin{itemize}
  \item $||v||:=\displaystyle\sum_{i=1}^m |v_i|$
  \item $x(\cdot)$ --- абсолютно непрерывная вектор-функция
    (траектория), $x(t)\in {\mathbb R}^n,$
  \item $v(\cdot)$ --- измеримая существенно ограниченная
    вектор-функция (управление),
  
  \item $V$ --- заданная величина интегрального ресурса управления
    $v$.
\end{itemize}

Пару функций $\bigl(x(\cdot),v(\cdot)\bigr),$ удовлетворяющих
системе \eqref{system_s}, будем называть процессом системы \eqref{system_s}.

Будем предполагать, что вектор-функция $f$ является функцией из
$L_{\infty}(T,\mathbb{R}^2)$, а $G$ задана следующим правилом:

\begin{equation*}
  G(x) = 
  \begin{pmatrix}
    a x_2+b & 0\\ 0 & c x_1 +d 
  \end{pmatrix}
\end{equation*}
     
Множество обычных траекторий системы \eqref{system_s} будет состоять
из функций, имеющих равномерно ограниченные полные вариации на отрезке
$T$. Следовательно, любая последовательность траекторий будет
содержать подпоследовательность, поточечно сходящуюся к некоторой
функции ограниченной вариации. Именно такие функции, являющиеся
поточечными пределами последовательностей обычных траекторий, в
дальнейшем будут называться разрывными (или обобщенными) решениями
системы \eqref{system_s}.

 Рассмотрим систему \eqref{system_d} соответствующую системе
\eqref{system_s}.

\begin{equation}
  \label{system_d}
  \begin{array}{l}
    dx(t)=f\big(t,x(t)\big)dt+G\big(t,x(t)\big)\pi(\mu), \quad
    x(0)=x_0, \\[8pt]
    \pi(\mu) \in \mathcal{W}(T)
  \end{array} 
\end{equation}

Здесь $T=[0,t_1]$ --- заданный отрезок времени, $x(\cdot)$ ---
непрерывная справа на $(0,t_1]$ функция ограниченной вариации, $x(t)
\in \mathbb{R}^2$. Решения системы \eqref{system_d}, соответсвующие
управлению $\pi(\mu)$, понимаются как решения интегрального уравнения
с мерой
\begin{equation*}
  x(t) = x_0 + \int_0^t f(t,x(t))dt + \int_0^t G(t,x(t)) \mu_c(dt) +
  \sum_{s \le t,\\s \in S_d(\mu)} (z_s(d_s) - x(s-)), \quad t \in (0,t_1]
\end{equation*}
где для каждого $s \in S_d(\mu)$ функция $z_s(\cdot)$ --- решение
дифференциального уравнения
\begin{equation*}
  \frac{dz_s(\tau)}{d\tau} = G(s,z_s(\tau))\omega_s(\tau), \quad z_s(0)=x(s-)
\end{equation*}

Таким образом, при расширении системы \eqref{system_s} и переходе к
соответствующей импульсной системе \eqref{system_d} мы к обычным
(абсолютно непрерывным) траекториям добавляем все частичные
поточечные пределы последовательностей обычных
траекторий. Полученное множество будем называть множеством
траекторий импульсной системы, соответствующей \eqref{system_s} и будем
обозначать импульсную систему символом \eqref{system_d} Поясним,
что если в системе \eqref{system_d} траектории могут иметь скачки,
то их обобщенные производные будут содержать дельтообразные
составляющие --- импульсы; следовательно, импульсы появятся в
соответствующих управлениях, отсюда и названия импульсной системы и
импульсного управления.

Приведем определение траекторий импульсной системы \eqref{system_d}

{\bf Определение 1.}
  {\it Функция ограниченной вариации $x(\cdot)$ называется траекторией
  импульсной системы \eqref{system_d} (или обобщенным (разрывным)
  решением системы \eqref{system_s}), если найдется такая последовательность
  траекторий системы \eqref{system_s} $\bigl\{x_k(\cdot)\bigr\}$, что выполняется
  условие 
\begin{equation*}
  x_k(t)\to x(t) \quad  \forall \, t\in [0,T].
\end{equation*}}

Определим \emph{множество достижимости} в момент  времени $t=t_1$
импульсной системы \eqref{system_d}. Обозначим это множество через 
$ {\mathcal R}_V(t_1)$. Это множество из  пространства ${\mathbb R}^n$
(конечномерное  множество), оно состоит из точек $x_b,$ в  которые в
момент $t=t_1$ попадают траектории  системы \eqref{system_d}, выходящие
в начальный  момент из начальной точки $x_0$; точнее
\begin{equation*} 
  {\mathcal R}_V(t_1):=\left\{ x(t_1+) \ \big| \
    x(\cdot) \mbox{ -- траектория системы } ({\mathcal D}) \mbox{ в
      смысле опр. 1} \right\}.
\end{equation*}

Укажем важные свойства множества достижимости \cite{ZS1991, SS2010}:

\begin{itemize}
\item[ 1)] Множество достижимости ${\mathcal R}_V(t_1)$ является
компактным множеством в ${\mathbb R}^n$.
\item[ 2)] Множество достижимости ${\mathcal R}_V(t_1)$ является связным
множеством в ${\mathbb R}^n,$ т.е. его нельзя представить в виде
объединения двух непустых непересекающихся подмножеств.
\item[ 3)] Множество достижимости ${\mathcal R}_V(t_1)$ непрерывно
зависит от величины интегрального ресурса управления $V$. Напомним,
что множество ${\mathcal R}_a(t_1)$ непрерывно зависит от параметра $a$
при $a=V,$ если
\begin{equation*} 
  \lim_{a\to V} \rho\Big( {\mathcal R}_a(t_1) , {\mathcal
    R}_V(t_1) \Big) =0 ,
\end{equation*} где $\rho(A,B)$ --- хаусдорфово расстояние между двумя компактными
множествами $A$ и $B$, по определению
\begin{equation}
  \label{eq:hausd_dist}
  \rho(A,B):=\max\bigl\{ \sup\limits_{y\in
A}\inf\limits_{z\in B} |y-z| ; \sup\limits_{z\in B}\inf\limits_{y\in
A} |y-z| \bigr\}
\end{equation}
\end{itemize}

Свойства, перечисленные выше, будут очень важны при численном построении внутренних
аппроксимаций множества достижимости.  В частности, благодаря свойству 3)
можно приближенно рассматривать множество достижимости как
некоторое конечное объединение множеств достижимости вспомогательной
системы.



%%% Local Variables:
%%% mode: latex
%%% TeX-master: "eikonal_solver"
%%% End:
