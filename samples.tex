\section{Примеры}
\label{sec:samples}

Разберем примеры работы алгоритма на двух примерах. Но для начала
введем несколько дополнительных сущностей. Первая из которых это как
на практике будет выглядеть проверка условия корректности Фробениуса,
о котором упоминалось в \eqref{sec:csdisttrack}. Для этого введем
понятие скобки Ли (коммутатора) двух гладких вектор-функций $\varphi(x)$,
$\psi(x)$ одной и той же размерности $n$, являющихся столбцами матрицы $G(x)$

Пусть $\varphi_x$, $\psi_x$ --- $n \times n$ матрицы частных
производных функций $\varphi$ и $\psi$, а $x$ --- вектор размерности $n$. В примерах рассматриваем
матрицу $G(x)$ размерности $n=2$. В этом случае функция $\varphi_x$ к
примеру будет выглядеть так:
\begin{equation*}
  \begin{pmatrix}
    \frac{d \varphi_1}{d x_1} & \frac{d \varphi_1}{d x_2} \\
    \frac{d \varphi_2}{d x_1} & \frac{d \varphi_2}{d x_2}
  \end{pmatrix}(x)
\end{equation*}
Матрица $\psi_x$ --- выглядит аналогично. Тогда скобка Ли
$[\varphi,\psi]$ функций $\varphi$, $\psi$ определяется равенством.
\begin{equation*}
  [\varphi, \psi](x) = \varphi_x(x) \psi(x) - \psi_x(x) \varphi(x)
\end{equation*}
Очевидно, что она антикоммутативна т.е. $[\varphi, \psi] = - [\psi,
\varphi]$.

При условии того, что мы используем матрицу $2 \times 2$, мы будем
говорить, что рассматриваемая управляемая система удовлетворяет
условию корректности Фробениуса, если выполняется равенство

\begin{equation}
  \label{eq:1}
  [\varphi,\psi](x) \equiv 0.
\end{equation}

Второе о чем стоит поговорить это способ, благодаря которому мы
оцениваем порядок точности полученного результата --- между двумя множествами
вычисляется расстояние Хаусдорфа \eqref{eq:hausd_dist}.

\subsection{Пример системы без дрейфа}
\label{sec:snwd}

Рассмотрим систему 

\begin{equation*}
  \begin{aligned}[b]
    &\dot{x_1}(t) = (1 -x_2(t))v_1(t), & x_1(0)=0\\
    &\dot{x_2}(t) = (1-x_1(t))v_2(t), & x_2(0) = 0\\[8pt]
    &v_1(t) \ge 0, v_2(t) \ge 0 \\
    &\int_{0}^{1} |v_1(t) + v_2(t)| dt \le V
  \end{aligned}
\end{equation*}


Стоит отметить, что для этой системы не выполнено условие корректности
Фробениуса (равенства нулю коммутатора). Покажем это. Возмем скобку
Ли (коммутатор) от двух гладких вектор-функций $\varphi$ и $\psi$,
являющимися столбцами матрицы

\begin{equation*}
  \begin{aligned}[b]
    G(x) = 
    \begin{pmatrix}
      \varphi & \psi
    \end{pmatrix},
    &
    \varphi =
    \begin{pmatrix}
      \varphi_1\\ \varphi_2
    \end{pmatrix},
    &
    \psi =
    \begin{pmatrix}
      \psi_1 \\ \psi_2
    \end{pmatrix}
  \end{aligned}
\end{equation*}


Скобка Ли тогда примет иметь вид ---

\begin{equation*}
  [\varphi,\psi] = \varphi_x \psi - \psi_x \varphi = 
  \begin{pmatrix}
    \frac{d \varphi_1}{d x_1} & \frac{d \varphi_1}{d x_2} \\
    \frac{d \varphi_2}{d x_1} & \frac{d \varphi_2}{d x_2}
  \end{pmatrix}
  \begin{pmatrix}
    \psi_1 \\ \psi_2
  \end{pmatrix}
  -
  \begin{pmatrix}
    \frac{d \psi_1}{d x_1} & \frac{d \psi_1}{d x_2} \\
    \frac{d \psi_2}{d x_1} & \frac{d \psi_2}{d x_2}
  \end{pmatrix}
  \begin{pmatrix}
    \varphi_1 \\ \varphi_2
  \end{pmatrix}
\end{equation*}

В этом примере матрица $G(x)$ имеет вид 

\begin{equation*}
  G(x) = 
  \begin{pmatrix}
    (1-x_2) & 0 \\
    0 & (1-x_1)
  \end{pmatrix}
\end{equation*}
В результмате несложных расчетов можно убедится, что условие корректности здесь не выполняется:
\begin{equation*}
  [\varphi,\psi] = 
  \begin{pmatrix}
    x_1 - 1\\
    x_2 - 1
  \end{pmatrix}
  \neq 0
\end{equation*}


Для этой системы была аналитически вычислена граница множества
достижимости в работе на основе работы \cite{AVS2016} Для разных
значений $V$ граница опиcывается немного по разному, но общий вид
формул отсается неизменным. На графиках красным цветом обозначены
линии, являющиеся границей.

%TODO: Нужны ли формулы?
На Рис.~\ref{fig:v11}-Рис.~\ref{fig:v72} представлены множества достижимости при различных
значениях ресурса $V$ и разных шагах сетки
\begin{figure}[h]
\centering
\begin{minipage}[h]{0.47\linewidth}
  \noindent \hfil
  \includegraphics[width=1\linewidth]{img/figure_v_1_h_01.png}
  \hfil \caption{Множество достижимости для системы с V=1, h=0.2}
  \label{fig:v11}
\end{minipage}
\hfill
\begin{minipage}[h]{0.47\linewidth}
  \noindent \hfil
  \includegraphics[width=1\linewidth]{img/figure_v_1_h_002.png}
  \hfil \caption{Множество достижимости для системы с V=1, h=0.02}
  \label{fig:v12}
\end{minipage}
\end{figure}
%\vfill
\begin{figure}[h]
\begin{minipage}[h]{0.47\linewidth}
  \noindent \hfil
  \includegraphics[width=1\linewidth]{img/figure_v_2_h_02.png}
  \hfil \caption{Множество достижимости для системы с V=2, h=0.2}
  \label{fig:v21}
\end{minipage}
\hfill
\begin{minipage}[h]{0.47\linewidth}
  \noindent \hfil
  \includegraphics[width=1\linewidth]{img/figure_v_2_h_002.png}
  \hfil \caption{Множество достижимости для системы с V=2, h=0.02}
  \label{fig:v22}
\end{minipage}
\vfill
\begin{minipage}[h]{0.47\linewidth}
  \noindent \hfil
  \includegraphics[width=1\linewidth]{img/figure_v_7_h_07.png}
  \hfil \caption{Множество достижимости для системы с V=7, h=0.7}
  \label{fig:v71}
\end{minipage}
\hfill
\begin{minipage}[h]{0.47\linewidth}
  \noindent \hfil
  \includegraphics[width=1\linewidth]{img/figure_v_7_h_014.png}
  \hfil \caption{Множество достижимости для системы с V=7, h=0.14}
  \label{fig:v72}
\end{minipage}
\end{figure}

Для различных значений ресурса $V$ и шага сетки $h$ мы вычислили
расстояние хаусдорфа между теоретичесской оценкой множества
достижимости и тем, что получилось в результате работы алгоритма

\begin{table}[f]
  \centering
  \begin{tabular}{|*{3}{c|}}
    $V$&$h$&$\rho(A,B)$\\ \hline
    1&0.2&0.19357953184\\
    1&0.1&0.0989472244814\\
    1&0.04&0.04\\
    1&0.02&0.02\\
    1&0.01&0.01\\
    2&0.4&0.335023013171\\
    2&0.2&0.193935429287\\
    2&0.08&0.080131031276\\
    2&0.04&0.04\\
    2&0.02&0.02\\
    3&0.6&0.6\\
    3&0.3&0.29644979071\\
    3&0.12&0.130019551125\\
    3&0.06&0.0659015266245\\
    3&0.03&0.0309661912566\\
    5&1.0&1.14082938538\\
    5&0.5&0.5\\
    5&0.2&0.205052066625\\
    5&0.1&0.0985691231869\\
    5&0.05&0.0470656986497\\
    7&1.4&1.89238926658\\
    7&0.7&1.0191890624\\
    7&0.28&0.363069883153\\
    7&0.14&0.139983207703\\
    7&0.07&0.0724479673101\\
  \end{tabular}
  \caption{Расстояние Хаусдорфа в зависимости от шага сетки для
    различных значений ресурса}
  \label{tab:hsd_ndft}
\end{table}

\pagebreak
\subsection{Пример системы с дрейфом}
\label{sec:swd}

Интегральные кривые этого примера при нулевом управлении
представляют собой окружности с центром в начале координат, причем
фазовая скорость по модулю всюду равена 1. Как и в предыдущем примере,
управления неотрицательны.

\begin{equation*}
  \begin{aligned}[b]
    &\dot{x_1}(t) = \frac{x_2}{\left\|x\right\|} + x_1v_1, &x_1(0) =
    0,\quad v_1(t) \ge 0;\\
    &\dot{x_2}(t) = \frac{-x_1}{\left\|x\right\|} + x_2v_2, &x_2(0)
    = 0,\quad v_2(t) \ge 0.\\
    &\int\limits_0^{13} |v_1(t)| + |v_2(t)| dt \le 1
  \end{aligned}
\end{equation*}
Здесь $\left\| \cdot \right\|$ --- евклидова норма.

Для этого примера условие корректности
Фробениуса выполняется. 
В этом примере матрица $G(x)$ имеет вид 

\begin{equation*}
  G(x) = 
  \begin{pmatrix}
    x_1 & 0 \\
    0 & x_2
  \end{pmatrix}
\end{equation*}
И нетрудно убедиться, что здесь   $[\varphi,\psi] \equiv 0$


\begin{figure}[h]
\centering
  \noindent \hfil
  \includegraphics[width=1\linewidth]{img/figure_d_h_001_ht_01}
  \hfil \caption{Множество достижимости для системы с V=1, h=0.02}
  \label{fig:v1h0.02}
\end{figure}


%%% Local Variables:
%%% mode: latex
%%% TeX-master: "rs-ids"
%%% End:
