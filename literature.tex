\pagebreak
\begin{thebibliography}{99}
  \addcontentsline{toc}{section}{Список литературы}


\bibitem{KL2011}{А. Л. Казаков, А. А. Лемперт}, “Об одном подходе к
  решению задач оптимизации, возникающих в транспортной логистике”,
  Автомат. и телемех., 2011, № 7, 50–57; Autom. Remote Control, 72:7
  (2011), 1398–1404

\bibitem{KLN2016}{А. Л. Казаков, А. А. Лемперт, Г. Л. Нгуен}, “Об одном
  алгоритме построения упаковки конгруэнтных кругов в неодносвязное
  множество с неевклидовой метрикой”, Выч. мет. программирование, 17:2
  (2016), 177–188

\bibitem{G1956}{С. К. Годунов}, “Разностный метод численного расчета
  разрывных решений уравнений гидродинамики”, Матем. сб., 47(89):3
  (1959), 271–306
  
\bibitem{D2003} {В. А. Дыхта, О. Н. Самсонюк}, Оптимальное импульсное
  управление с приложениями, М.: Физматлит, (2003).

\bibitem{V2005} {О. И. Вдовина, А. Н. Сесекин}, ``Численное построение
  областей достижимости для систем с импульсным управлением'' Труды
  Института математики и механики УрО РАН, 11, No.~1, 65-73 (2005).

\bibitem{L2016} {П. Д. Лебедев, А. А. Успенский}, “Построение функции
  оптимального результата и рассеивающих линий в задачах
  быстродействия с невыпуклым целевым множеством”, Тр. ИММ УрО РАН,
  22, № 2, 2016, 188–198

  
\bibitem{P2008} {П. Д. Лебедев, А. А. Успенский, В. Н. Ушаков},
  “Построение минимаксного решения уравнения типа эйконала”, Тр. ИММ
  УрО РАН, 14, № 2, 2008, 182–191; Proc. Steklov Inst. Math. (Suppl.),
  263, suppl. 2 (2008), S191–S201
  
\bibitem{E2014}{Е. Д. Москаленский}, “О нахождении точных решений двумерного
  уравнения эйконала для случая, когда фронт волны, распространяющейся
  в среде, является окружностью”, Сиб. журн. вычисл. матем., 17:4
  (2014), 363–372; Num. Anal. Appl., 7:4 (2014), 304–313

\bibitem{B2006} {А. В. Боровский} Группы эквивалентности уравнений
  эйконала и классы эквивалентных уравнений //Новосибирский
  государственный университет. – Новосибирский государственный
  университет, 2006.

\bibitem{B2014} {А. В. Боровских}, “Уравнение эйконала для анизотропной
  среды”, Тр. сем. им. И. Г. Петровского, 29, Изд-во Моск. ун-та, М.,
  2013, 162–229; J. Math. Sci. (N. Y.), 197:2 (2014), 248–289

\bibitem{I2005} {Д. И. Иванов, И. Э. Иванов, И. А. Крюков}, “Алгоритмы
  приближенного решения некоторых задач прикладной геометрии,
  основанные на уравнении типа Гамильтона–Якоби”, Ж. вычисл. матем. и
  матем. физ., 45:8 (2005), 1345–1358; Comput. Math. Math. Phys., 45:8
  (2005), 1297–1310

\bibitem{N2015} {Дучков А. А. и др.} Параллельный алгоритм решения
  уравнения эйконала для трехмерных задач сейсморазведки
  //Новосибирский государственный университет. – Новосибирский
  государственный университет, 2015.
  
\bibitem{AVS2016} {Д. В. Апанович, В. А. Воронов, О. Н. Самсонюк},
  Построение множества достижимости двумерной импульсной управляемой
  системы с билинейной структурой// Изв. Иркутского
  гос. ун-та. Сер. Математика, 15 (2016), 3–16

\bibitem{AV2015_1} {Д. В. Апанович, В. А. Воронов}, Численная аппроксимация
  множеств достижимости нелинейных импульсных управляемых систем. //
  Теория управления и математическое моделирование: Тезисы докладов
  Всероссийской конференции с международным участием, посвященной
  памяти профессора Н. В. Азбелева и профессора Е. Л. Тонкова (Ижевск,
  Россия, 9–11 июня 2015 г.). Ижевск: Изд-во «Удмуртский университет»,
  2015. С. 24-25.

\bibitem{AV2015_2} {Д. В. Апанович, В. А. Воронов}, Численная аппроксимация
  неодносвязного множества достижимости нелинейной импульсной
  управляемой системы. // Тезисы докладов III Российско-монгольской
  конференции молодых ученых по математическому моделированию,
  вычислительно-информационным технологиям и управлению (Иркутск
  (Россия) – Ханх (Монголия), 23 июня – 30 июня 2015 г.). – Иркутск:
  Научно-организационный отдел ИДСТУ СО РАН, 2015. С. 17.
  
\bibitem{S1999} {J. A. Sethian}, Level Set Methods and Fast Marching
  Methods: evolving interfaces in computational geometry, computer
  vision and matherial science. — 2nd Ed. — Cambridge Univ. Press,
  1999

\bibitem{B2006}{P. Berczynski et al.} Diffraction of a Gaussian beam
  in a three-dimensional smoothly inhomogeneous medium: an
  eikonal-based complex geometrical-optics approach //JOSA A. –
  2006. – Т. 23. – №. 6. – С. 1442-1451.

\bibitem{W1969} {A. Walther} Lenses, wave optics, and eikonal
  functions //JOSA. – 1969. – Т. 59. – №. 10. – С. 1325-1331.
  
\bibitem{AV2003}{J. A. Sethian, A. Vladimirsky} Ordered upwind methods
  for static Hamilton--Jacobi equations: Theory and algorithms //SIAM
  Journal on Numerical Analysis. – 2003. – Т. 41. – №. 1. –
  С. 325-363.
  
\bibitem{V1983} {M. G. Crandall, P. L. Lions}, Viscosity solutions of
  Hamilton-Jacobi equations //Transactions of the American
  Mathematical Society. – 1983. – Т. 277. – №. 1. – С. 1-42.

\bibitem {V1984} {M. G. Crandall, L. C. Evans, P. L. Lions}, Some
  properties of viscosity solutions of Hamilton-Jacobi equations
  //Transactions of the American Mathematical Society. – 1984. –
  Т. 282. – №. 2. – С. 487-502.

  
\bibitem{F2005} {H. K. Zhao}, A fast sweeping method for eikonal
  equations //Mathematics of computation. – 2005. – Т. 74. – №. 250. –
  С. 603-627.

\bibitem{K2017} {Кормен, Томас Х, и др.}, Алгоритмы: построение и
  анализ, 3-е издание / Т.Х. Кормен, Ч.И. Лейзерсон, Р.Л. Ривест,
  К. Штайн. – М. : Вильямс, 2013. – 1328 с.

\bibitem{R1974} {A. K. Jain, L. Hong, and S. Pankanti},Random
  insertion into a priority queue structure //IEEE Transactions on
  Software Engineering. – 1975. – №. 3. – С. 292-298.


\bibitem{F1987} {M. Fredman and R. Tarjan}, Fibonacci heaps and their
  uses in improved network optimization algorithms //Journal of the
  ACM (JACM). – 1987. – Т. 34. – №. 3. – С. 596-615.

  
\bibitem{J2015}{Javier V. Gomez Gonzalez}, Fast Marching Methods in
  path and motion planning: improvements and high-level
  applications. – 2015.

\bibitem{C2013}{A. Capozzoli et al.} A comparison of Fast Marching,
  Fast Sweeping and Fast Iterative Methods for the solution of the
  eikonal equation //Telecommunications Forum (TELFOR), 2013 21st. –
  IEEE, 2013. – С. 685-688.
  
\bibitem{KOT2005} {C. Y. Kao, S. Osher, Y. H. Tsai} Fast Sweeping
  Methods for Static Hamilton--Jacobi Equations //SIAM journal on
  numerical analysis. – 2005. – Т. 42. – №. 6. – С. 2612-2632.

\bibitem{A2006}{O. Alvarez } et al. Convergence of a first order
  scheme for a non-local eikonal equation //Applied numerical
  mathematics. – 2006. – Т. 56. – №. 9. – С. 1136-1146.
  
\bibitem{FSA2007} {Jianliang Qian, Yong-Tao Zhang, Hong-Kai Zhao} A
  fast sweeping method for static convex Hamilton–Jacobi equations//
  Journal of Scientific Computing 31 (1), 237-271
  
\bibitem{FS2007} {Jianliang Qian, Yong-Tao Zhang, Hong-Kai Zhao} Fast
  sweeping methods for Eikonal equations on triangular meshes// SIAM
  Journal on Numerical Analysis, vol 45, no 1, pp. 596–615, 2007

\bibitem{C2017} {Б. Клеменс} Язык С в XXI веке. – Litres, 2017.

\bibitem{N1989}{A. Iserles} Numerical Recipes in C: The art of scientific
  computing. – 1989.
  
\bibitem{AP2016} {Э. С. Рэймонд} Искусство программирования для Unix. :
  Пер. с англ. --- М. ООО ``И.Д. Вильямс'', 2016. --- 544 с. : ил. ---
  Парал. тит. англ.

\bibitem{F2002} {R. J. Leveque}, Finite Volume Methods for Hyperbolic
  Problems, Cambridge University Press, 2002.

\bibitem{T2002} {А. В. Скворцов} Триангуляция Делоне и её применение. –
  Томск: Изд-во Том. ун-та, 2002. – 128 с.

\bibitem{SFS2009} {Emmanuel Prados, Olivier Faugeras}, Shape from
  Shading: a well-posed problem?. IEEE Conference on Computer Vision
  and Pattern Recognition, CVPR’05, Jun 2005, San Diego, United
  States. IEEE,2, pp.870-877, 2005

\bibitem{JDM2008}{Jean-Denis Durou,Maurizio Falcone, Manuela Sagona},
  Numerical methods for shape-from-shading: A new survey with
  benchmarks, Computer Vision and Image Understanding 109 (2008) 22–43
\end{thebibliography}


%%% Local Variables:
%%% mode: latex
%%% TeX-master: "eikonal_solver"
%%% End:
