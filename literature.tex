\pagebreak
\begin{thebibliography}{99}
  \addcontentsline{toc}{section}{Список литературы}

\bibitem{S1999} {J.A. Sethian.} Level Set Methods and Fast Marching Methods: evolving
  interfaces in computational geometry, computer vision and matherial
  science. — 2nd Ed. — Cambridge Univ. Press, 1999

\bibitem {V1984} {M.G. Crandall, L.C. Evans, P.L. Lions}, Some
  Properties of Viscosity Solutions of Hamilton-Jacobi Equations,
  Trans. Amer. Math. Soc. 282 (1984), pp. 487-502.

\bibitem{V1983} {M.G. Crandall,P.L. Lions}, Viscosity Solutions of
  Hamilton-Jacobi Equations, Trans. Amer. Math. Soc. 277 (1983),
  pp. 1-42.

\bibitem{F2005} {H.K.Zhao}, Fast Sweeping Method for Eikonal Equations, Math. Comp.,
74,(2005)

\bibitem{K2017} {Кормен, Томас Х, и др.}, Алгоритмы: построение и
  анализ, 3-е издание: Пер. с англ. -- М. : ООО
  ``И.Д. Вильямс'',2017. -- 1328 с. : ил. -- Парал. тит. англ.

\bibitem{R1974} {A. K. Jain, L. Hong, and S. Pankanti}, “Random insertion into a priority queue
structure,” tech. rep., Stanford Univeristy Reports, 1974.


\bibitem{F1987} {M. Fredman and R. Tarjan}, “Fibonacci Heaps and Their Uses in Improved
Network Optimization Algorithms,” Journal of the Association for Computing
Machinery, vol. 34, no. 3, pp. 596–615, 1987.

  
\bibitem{AVS2016} {Апанович Д. В.,Воронов В. А.,Самсонюк О. Н.},
  Построение множества достижимости двумерной импульсной управляемой
  системы с билинейной структурой// Изв. Иркутского
  гос. ун-та. Сер. Математика, 15 (2016), 3–16

\bibitem{AV2015_1} Апанович Д.В., Воронов В.А. Численная аппроксимация
  множеств достижимости нелинейных импульсных управляемых систем. //
  Теория управления и математическое моделирование: Тезисы докладов
  Всероссийской конференции с международным участием, посвященной
  памяти профессора Н. В. Азбелева и профессора Е. Л. Тонкова (Ижевск,
  Россия, 9–11 июня 2015 г.). Ижевск: Изд-во «Удмуртский университет»,
  2015. С. 24-25.

\bibitem{AV2015_2} Апанович Д.В., Воронов В.А. Численная аппроксимация
  неодносвязного множества достижимости нелинейной импульсной
  управляемой системы. //  Тезисы докладов III Российско-монгольской
  конференции молодых ученых по математическому моделированию,
  вычислительно-информационным технологиям и управлению (Иркутск
  (Россия) – Ханх (Монголия), 23 июня – 30 июня 2015 г.). – Иркутск:
  Научно-организационный отдел ИДСТУ СО РАН, 2015. С. 17.

\bibitem{FSA2007} {Jianliang Qian, Yong-Tao Zhang, Hong-Kai Zhao} A
  fast sweeping method for static convex Hamilton–Jacobi equations//
  Journal of Scientific Computing 31 (1), 237-271
  
\bibitem{FS2007} {Jianliang Qian, Yong-Tao Zhang, Hong-Kai Zhao} Fast
  sweeping methods for Eikonal equations on triangular meshes// SIAM
  Journal on Numerical Analysis, vol 45, no 1, pp. 596–615, 2007
  

\bibitem{F2002} {R.J. Leveque}, Finite Volume Methods for Hyperbolic Problems, Cambridge
University Press, 2002.

\bibitem{T2002} {Скворцов А.В.} Триангуляция Делоне и её применение. –
  Томск: Изд-во Том. ун-та, 2002. – 128 с.

\bibitem{SFS2009} {Emmanuel Prados, Olivier Faugeras}, Shape from Shading: a well-posed problem?. IEEE Conference
on Computer Vision and Pattern Recognition, CVPR’05, Jun 2005, San Diego, United States. IEEE,
2, pp.870-877, 2005, <10.1109/CVPR.2005.319>. <inria-00394230>

  
\end{thebibliography}


%%% Local Variables:
%%% mode: latex
%%% TeX-master: "eikonal_solver"
%%% End:
