\pagebreak
\begin{thebibliography}{99}
  \addcontentsline{toc}{section}{Список литературы}

\bibitem{DS2011} { Dykhta~V.A., Samsonyuk~O.N.}  Some applications of
  Hamilton-Jacobi inequalities for classical and impulsive optimal
  control problems // European Journal of Control. Nonsmooth analysis,
  Control and Optimization. 2011. Vol.17. Pp. 55--69.

\bibitem{ZS1991} { Завалищин С.Т., Сесекин А.Н.}  {Импульсные
    процессы: модели и приложения}.  М.: Наука, 1991.

\bibitem{MR2005} { Миллер Б.М., Рубинович Е.Я.}  { Оптимизация
    динамических систем с импульсными управлениями}.  М.: Наука, 2005.

\bibitem{DS2003} { Дыхта В.А., Самсонюк О.Н.}  Оптимальное импульсное
  управление с приложениями.  М.: Физматлит, 2003.

\bibitem{SS2010} { Самсонюк~О.Н., Сесекин~А.Н.} Оценки и свойства
  интегральных воронок траекторий нелинейных импульсных систем //
  Тез. докл. II Международной школы-семинара «Нелинейный анализ и
  экстремальные задачи». Иркутск, ИДСТУ СО РАН, 28 июня -- 4 июля 2010
  г. 2010. С. 64--65.

\bibitem{BS2005} Вдовина О.И., Сесекин А.Н. Численное построение
  областей достижимости для систем с импульсными управлениями //
  Тр. Ин-та математики и механики УрО РАН.  2005. Т. 11, № 1.

\bibitem{G1997} Гурман В.И. Принцип расширения в задачах
  управления. 2-е изд., перераб. и доп. М.:Физматлит, 1997, 288 с.

\bibitem{M1993} Миллер Б.М. Метод разрывной замены времени в задачах
  оптимального управления импульсными и дискретно-непрерывными
  системами // Автоматика и телемеханика. 1993. № 12. С. 3--32.

\bibitem{MR1995} Motta M., Rampazzo F. Space-time trajectories of
  nonlinear systems driven by ordinary and impulsive controls //
  Differential Integral Equations. 1995. Vol. 8.  Pp. 269-288.

\bibitem{MS1999} Motta M., Sartori C. Discontinuous solutions to
  unbounded differential inclusions under state constraints.
  Applications to optimal control problems // Set-Valued
  Analysis. 1999. Vol. 7. Pp. 295-322.

\bibitem{WZ2007} Wolenski P.R., Zabic S.  A Sampling Method and
  Approximation Results for Impulsive Systems // SIAM J. Control
  Optim., 2007, Vol. 46. Pp. 983-998.

\bibitem{AVS2016} {Д. В. Апанович, В. А. Воронов, О. Н. Самсонюк},
  Построение множества достижимости двумерной импульсной управляемой
  системы с билинейной структурой// Изв. Иркутского
  гос. ун-та. Сер. Математика, 15 (2016), 3–16

\bibitem{DS2000} {Дыхта В. А., Самсонюк О.Н.} Оптимальное импульсное
  управление с приложениями. — М.: ФИЗМАТ ЛИТ, 2000. — 256 с. — ISBN
  5-9221-0097-1.

\bibitem{S1999} {J.A. Sethian.} Level Set Methods and Fast Marching Methods: evolving
interfaces in computational geometry, computer vision and matherial
science. — 2nd Ed. — Cambridge Univ. Press, 1999

\end{thebibliography}


%%% Local Variables:
%%% mode: latex
%%% TeX-master: "rs-ids"
%%% End:
