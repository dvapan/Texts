\pagebreak
\begin{thebibliography}{99}
  \addcontentsline{toc}{section}{Список литературы}


\bibitem{KL2011}Казаков А. Л., Лемперт А. А. Об одном подходе к
  решению задач оптимизации, возникающих в транспортной логистике
  //Автоматика и телемеханика. – 2011. – №. 7. – С. 50-57.

\bibitem{KLN2016}Казаков А. Л., Лемперт А. А., Нгуен Г. Л. Об одном
  алгоритме построения упаковки конгруэнтных кругов в неодносвязное
  множество с неевклидовой метрикой //Вычислительные методы и
  программирование. – 2016. – Т. 17. – №. 2. – С. 177-188.

\bibitem{G1956}Годунов С. К. Разностный метод численного расчета
  разрывных решений уравнений гидродинамики //Математический
  сборник. – 1959. – Т. 47. – №. 3. – С. 271-306.
  
\bibitem{D2003}Дыхта В. А., Самсонюк О. Н. Оптимальное импульсное
  управление с приложениями. – М. : Физматлит, 2003.

\bibitem{V2005} Вдовина О. И., Сесекин А. Н. Численное построение
  областей достижимости для систем с импульсным управлением //Труды
  Института математики и механики УрО РАН. – 2005. – Т. 11. – №. 1. –
  С. 65-73.

\bibitem{L2016} Лебедев П. Д., Успенский А. А. Построение функции
  оптимального результата и рассеивающих линий в задачах
  быстродействия с невыпуклым целевым множеством //Труды Института
  математики и механики УрО РАН. – 2016. – Т. 22. – №. 2. –
  С. 188-198.

  
\bibitem{P2008} Лебедев П. Д., Успенский А. А., Ушаков
  В. Н. Построение минимаксного решения уравнения типа эйконала
  //Труды Института математики и механики УрО РАН. – 2008. – Т. 14. –
  №. 2. – С. 182-191.
  
\bibitem{E2014}Москаленский Е. Д. О нахождении точных решений
  двумерного уравнения эйконала для случая, когда фронт волны,
  распространяющейся в среде, является окружностью //Сибирский журнал
  вычислительной математики. – 2014. – Т. 17. – №. 4. – С. 363-372.

\bibitem{B2006} Боровских А. В. Группы эквивалентности уравнений
  эйконала и классы эквивалентных уравнений //Новосибирский
  государственный университет. – Новосибирский государственный
  университет, 2006.

\bibitem{B2014} Боровских А. В. Уравнение эйконала для анизотропной
  среды //Труды семинара имени ИГ Петровского. – 2013. – Т. 29. –
  №. 0. – С. 162-229.

\bibitem{I2005} Иванов Д. И., Иванов И. Э., Крюков И. А. Алгоритмы
  приближенного решения некоторых задач прикладной геометрии,
  основанные на уравнении типа Гамильтона–Якоби //Журнал
  вычислительной математики и математической физики. – 2005. –
  Т. 45. – №. 8. – С. 1345-1358.

\bibitem{N2015} Дучков А. А. и др. Параллельный алгоритм решения
  уравнения эйконала для трехмерных задач сейсморазведки
  //Новосибирский государственный университет. – Новосибирский
  государственный университет, 2015.
  
\bibitem{AVS2016} Апанович Д. В., Воронов В. А., Самсонюк
  О. Н. Построение множества достижимости двумерной импульсной
  управляемой системы с билинейной структурой //Известия Иркутского
  государственного университета. Серия: Математика. – 2016. – Т. 15.

\bibitem{AV2015_1} Апанович Д. В., Воронов В. А. Численная аппроксимация
  множеств достижимости нелинейных импульсных управляемых систем. //
  Теория управления и математическое моделирование: Тезисы докладов
  Всероссийской конференции с международным участием, посвященной
  памяти профессора Н. В. Азбелева и профессора Е. Л. Тонкова (Ижевск,
  Россия, 9–11 июня 2015 г.). Ижевск: Изд-во «Удмуртский университет»,
  2015. С. 24-25.

\bibitem{AV2015_2} Апанович Д. В., Воронов В. А. Численная
  аппроксимация неодносвязного множества достижимости нелинейной
  импульсной управляемой системы. // Тезисы докладов III
  Российско-монгольской конференции молодых ученых по математическому
  моделированию, вычислительно-информационным технологиям и управлению
  (Иркутск (Россия) – Ханх (Монголия), 23 июня – 30 июня 2015 г.). –
  Иркутск: Научно-организационный отдел ИДСТУ СО РАН, 2015. С. 17.
  
\bibitem{S1999} Sethian J. A. Level set methods and fast marching
  methods: evolving interfaces in computational geometry, fluid
  mechanics, computer vision, and materials science. – Cambridge
  university press, 1999. – Т. 3.

\bibitem{B2006}Berczynski P. et al. Diffraction of a Gaussian beam in
  a three-dimensional smoothly inhomogeneous medium: an eikonal-based
  complex geometrical-optics approach //JOSA A. – 2006. – Т. 23. –
  №. 6. – С. 1442-1451.

\bibitem{W1969} Walther A. Lenses, wave optics, and eikonal functions
  //JOSA. – 1969. – Т. 59. – №. 10. – С. 1325-1331.
  
\bibitem{AV2003} Sethian J. A., Vladimirsky A. Ordered upwind methods
  for static Hamilton--Jacobi equations: Theory and algorithms //SIAM
  Journal on Numerical Analysis. – 2003. – Т. 41. – №. 1. –
  С. 325-363.
  
\bibitem{V1983} Crandall M. G., Lions P. L. Viscosity solutions of
  Hamilton-Jacobi equations //Transactions of the American
  Mathematical Society. – 1983. – Т. 277. – №. 1. – С. 1-42.

\bibitem {V1984} Crandall M. G., Evans L. C., Lions P. L. Some
  properties of viscosity solutions of Hamilton-Jacobi equations
  //Transactions of the American Mathematical Society. – 1984. –
  Т. 282. – №. 2. – С. 487-502.

  
\bibitem{F2005} Zhao H. K. A fast sweeping method for eikonal
  equations //Mathematics of computation. – 2005. – Т. 74. – №. 250. –
  С. 603-627.

\bibitem{K2017} Кормен Т. и др. Алгоритмы. Построение и анализ:[пер. с
  англ.]. – Издательский дом Вильямс, 2017.

\bibitem{R1974} Porter T., Simon I. Random insertion into a priority
  queue structure //IEEE Transactions on Software Engineering. –
  1975. – №. 3. – С. 292-298.


\bibitem{F1987} Fredman M. L., Tarjan R. E. Fibonacci heaps and their
  uses in improved network optimization algorithms //Journal of the
  ACM (JACM). – 1987. – Т. 34. – №. 3. – С. 596-615.

  
\bibitem{J2015}Gómez González J. V. Fast Marching Methods in path and
  motion planning: improvements and high-level applications. – 2015.
  
\bibitem{C2013}Capozzoli A. et al. A comparison of Fast Marching, Fast
  Sweeping and Fast Iterative Methods for the solution of the eikonal
  equation //Telecommunications Forum (TELFOR), 2013 21st. – IEEE,
  2013. – С. 685-688.
  
\bibitem{KOT2005} Kao C. Y., Osher S., Tsai Y. H. Fast Sweeping
  Methods for Static Hamilton--Jacobi Equations //SIAM journal on
  numerical analysis. – 2005. – Т. 42. – №. 6. – С. 2612-2632.

\bibitem{A2006}Alvarez O. et al. Convergence of a first order scheme
  for a non-local eikonal equation //Applied numerical mathematics. –
  2006. – Т. 56. – №. 9. – С. 1136-1146.
  
\bibitem{FSA2007}Qian J., Zhang Y. T., Zhao H. K. A fast sweeping
  method for static convex Hamilton–Jacobi equations //Journal of
  Scientific Computing. – 2007. – Т. 31. – №. 1. – С. 237-271.
  
\bibitem{FS2007} Qian J., Zhang Y. T., Zhao H. K. Fast sweeping
  methods for Eikonal equations on triangular meshes //SIAM Journal on
  Numerical Analysis. – 2007. – Т. 45. – №. 1. – С. 83-107.

\bibitem{C2017} Клеменс Б. Язык С в XXI веке. – Litres, 2017.

\bibitem{N1989}Seiler M. C., Seiler F. A. Numerical recipes in C: the
  art of scientific computing //Risk Analysis. – 1989. – Т. 9. –
  №. 3. – С. 415-416.
  
\bibitem{AP2016} Реймонд Э. C. Искусство программирования для Unix. –
  Издательский дом Вильямс, 2005.

\bibitem{F2002} LeVeque R. J. Finite volume methods for hyperbolic
  problems. – Cambridge university press, 2002. – Т. 31.

\bibitem{T2002} {А. В. Скворцов} Триангуляция Делоне и её применение. –
  Томск: Изд-во Том. ун-та, 2002.

\bibitem{SFS2009} Prados E., Faugeras O. Shape from shading: a
  well-posed problem? //Computer Vision and Pattern Recognition,
  2005. CVPR 2005. IEEE Computer Society Conference on. – IEEE,
  2005. – Т. 2. – С. 870-877.
  
\bibitem{JDM2008}Durou J. D., Falcone M., Sagona M. Numerical methods
  for shape-from-shading: A new survey with benchmarks //Computer
  Vision and Image Understanding. – 2008. – Т. 109. – №. 1. –
  С. 22-43.
\end{thebibliography}


%%% Local Variables:
%%% mode: latex
%%% TeX-master: "eikonal_solver"
%%% End:
