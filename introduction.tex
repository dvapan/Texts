\section*{Введение}
\addcontentsline{toc}{section}{Введение}
\label{intro}

% Оценки множества достижимости управляемой системы, т.е. множества всех
% состояний, в которые система может быть переведена из заданной
% начальной точки за фиксированное время, играют важную роль как в
% теории динамических систем, так и в некоторых прикладных задачах.  В
% частности, в задаче оптимального управления с терминальным
% функционалом знание множества достижимости позволяет найти значение
% задачи и конечную точку оптимальной траектории как решение задачи
% математического программирования на множестве достижимости. Знание
% множества достижимости делает тривиальным вопрос о существовании
% допустимого управления в задаче с закрепленными концами (т.е. об
% управляемости пары состояний).  В том случае, если задана конечная
% точка траектории, аналогичным образом в обратном времени вычисляется
% множество всех состояний, из которых можно перейти в заданное.
% Поскольку множество достижимости, как правило, не может быть найдено
% точно, используются внутренние и внешние оценки, которые,
% соответственно, дают лишь приближенный ответ в перечисленных выше
% задачах.

% Оценка множества достижимости позволяет понять, как может вести себя
% система, замкнутая по обратной связи, при неизвестных внешних
% воздействиях (к примеру, действие ветра на самолет, управляемый
% автопилотом), и позволяет определить, не выходят ли значения
% переменных за грань допустимых.  Также можно привести в качестве
% примера задачу уклонения самолета от выпущенной в него ракеты ---
% здесь оценки множества достижимости характеризуют, с одной стороны,
% начальные состояния, при которых ракета попадет в самолет при любой
% его допустимой траектории; с другой стороны --- множество состояний,
% при которых самолет гарантированно уклонится от ракеты.

% Проблеме оценки множеств достижимости управляемых динамических систем,
% описываемых системой обыкновенных дифференциальных уравнений,
% посвящено большое количество работ.  Широко применяется аппарат
% эллипсоидального исчисления, позволяющий находить как внутренние, так
% и внешние оценки (А.Б. Куржанский, Ф.Л. Черноусько).  Известны также
% методы, в которых оценки строят в виде многогранников (в частности,
% параллелепипедов).  Принципиально другой подход основан на методе
% динамического программирования Р.Беллмана: множество достижимости
% рассматривается как поверхность уровня функции цены.

% В управляемых системах возможны ситуации, когда за короткий период
% времени происходит существенное изменение состояния. Данная ситуация
% возникает, например, в модели, описывающей многоимпульсные
% межорбитальные переходы спутника Земли; в некоторых экономических и
% медико-биологических моделях. В том случае, если временем, отведенным
% на быстрое изменение состояния, можно пренебречь, полученную
% управляемую систему называют импульсной. Задачи оптимального
% управления импульсными системами имеют свою специфику; это справедливо
% и для задачи построения оценок множества достижимости.  Метод
% разрывной замены времени позволяет свести построение множества
% достижимости для импульсной системы к постановке для системы с
% непрерывными траекториями и ограниченным управлением, но ограничения
% на управление при этом подходе не позволяют напрямую применить
% какой-либо из перечисленных выше методов.

% Существуют два основных подхода для численной аппроксимации множества
% достижимости импульсных управляемых систем. В первом случае
% исследуемую задачу "огрубляют", т.е. заменяют множество достижимости
% его оценкой (внешней и внутренней). Например, в статьях Т.Ф. Филиповой
% используются эллипсоидальные оценки \cite{FM2011}. Во втором случае
% делаются попытки построить множество достижимости на основании
% аппрокисмации допустимых управлений линейными комбинациями
% дельта-функций Дирака \cite{BS2005}.

% Основная цель выпускной работы состояла в изучении возможности и
% попытке реализовать алгоритмчисленного построения множества
% достижимости импульсной управляемой системы с билинейной
% структурой. Были реализованы два алгоритма: (1) перебор некоторого
% конечного множества допустимых управлений и (2) приближенное
% вычисление затраченного ресурса для точек сетки на основе метода
% динамического программирования. Второй алгоритм основан на результатах
% монографии Дж. Сейтиана \cite{S1999}, разработавшего так называемый
% Fast Marching Method для численного решения уравнения эйконала.
% Наиболее важным результатом выпускной работы стала реализация идеи
% динамического программирования, что позволило в разы увеличить
% скорость, с которой строится аппроксимация системы по сравнению с
% подходом в статье \cite{BS2005}.

%%% Local Variables:
%%% mode: latex
%%% TeX-master: "eikonal_solver"
%%% End:
