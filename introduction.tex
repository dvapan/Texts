\section*{Введение}
\addcontentsline{toc}{section}{Введение}
\label{intro}

Существует ряд прикладных задач, связанных с динамическими
управляемыми системами, когда необходимо понять как будет вести себя
система при различных управлениях. Для этого необходимо понять, как
устроено множество достижимости системы, т.е. множества таких точек
фазового пространства, в которые система может быть переведена из
начального сотояния за заданное время. Такая информация даст наиболее
полную информацию о возможностях системы.

Приведем несколько примеров зачем вообще нужно строить аппроксимации
множества достижимости. Знание множества достижимости позволяет найти
значение функционала задачи оптимального управления, как решение
задачи математического программирования на множестве
достижимости. Зная множество достижимости, можно ставить вопрос о
существовании допустимого управления. Оценка множества достижимости
позволяет понять, как ведет себя система при неизвестных внешних
воздействиях, когда управление задано как позиционное (к примеру
воздействие ветра на самолет во время посадки), позволит определить
--- не выходят ли его параметры за грань допустимых.  Так же можно
привести в пример задачу уклонения объекта, скажем вертолета от
выпущеной в него ракеты.

В управляемых системах возможны ситуации, когда за
очень короткий период времени происходит существенное изменение
некоторых параметров, описывающих функционирование системы.  Важные
примеры подобных ситуаций можно найти в математической экологии,
например, рациональное использование и охрана водных, земельных,
атмосферных, минеральных и энергетических ресурсов, аварийный разлив
нефти и моделирование распространения нефтяного пятна,  а также в
экономике, механике, ракетодинамике,  квантовой электронике (лазерное
излучение является импульсным по своей природе), робототехнике,
медиотерапии  и т.д. В частности, такие ситуации возникают при
моделировании реальных процессов, управление которыми осуществляется в
течение столь кратковременных промежутков, что их можно принимать как
мгновенные, а результаты воздействия приводят к быстрому изменению
процесса --- скачкам фазовой траектории моделируемой системы. Динамика
таких процессов в первом приближении может быть описана динамическими
системами  с разрывными траекториями и управлениями импульсного типа,
т.е. такими, при которых происходит резкое, скачкообразное изменение
состояния системы в отдельные моменты времени. 

Существуют два основных подхода для численной аппроксимации множества
достижимости импульсных управляемых систем. В первом случае
исследуемую задачу "огрубляют", т.е. заменяют множество достижимости
его оценкой (внешней и внутренней). Например в статьях Т.Ф. Филиповой
используются эллипсоидальные оценки \cite{FM2011}. Во втором случае
делаются попытки построить множество достижимости на основании
аппрокисмации допустимых управлений линейными комбинациями
дельта-функций Дирака \cite{BS2005}. Кроме этого существует два
подхода, разработанные для решения уравнения Эйконала Джеймсом
Сейтианом --- Level Set Method и Fast Marching Method \cite{S1999}.

Основная цель дипломной работы состояла в изучении возможности и
попытке реализовать алгоритм численного построения множества
достижимости импульсной управляемой системы билинейной
структуры. Наиболее важным реультатом стала реализация идеи заложенной
в методах предложенном Джеймсом Сейтианом \cite{S1999} для импульсной
системы, что позволило в разы увеличить скорость, с которой строится
множество достижимости, по сравнению с подходом в статье \cite{BS2005}



%%% Local Variables:
%%% mode: latex
%%% TeX-master: "rs-ids"
%%% End:
