\chapter*{Введение}
\addcontentsline{toc}{chapter}{Введение}
\label{intro}

Здесь порассуждаем о МД и зачем оно нам нужно 


Оценка множества достижимости позволяет понять, как ведет себя система
при неизвестных внешних воздействиях.


Знание множества достижимости дает максимальную информацию о
возможностях системы и помогает находить оптимальные управления для
различных функционалов. Необходимость  исследования свойств пучков
разрывных траекторий и их МД связана с тем, что в механике полета,
электрофизике, медицине и экономике часто встречаются динамические
объекты, допускающие разрывные траектории, порожденные импульсными
управлениями.


Существует ряд прикладных задач, в которых возможны ситуации, когда за
очень короткий период времени происходит существенное изменение
некоторых параметров, описывающих функционирование системы.  Важные
примеры подобных ситуаций можно найти в математической экологии,
например, рациональное использование и охрана водных, земельных,
атмосферных, минеральных и энергетических ресурсов, аварийный разлив
нефти и моделирование распространения нефтяного пятна,  а также в
экономике, механике, ракетодинамике,  квантовой электронике (лазерное
излучение является импульсным по своей природе), робототехнике,
медиотерапии  и т.д. В частности, такие ситуации возникают при
моделировании реальных процессов, управление которыми осуществляется в
течение столь кратковременных промежутков, что их можно принимать как
мгновенные, а результаты воздействия приводят к быстрому изменению
процесса --- скачкам фазовой траектории моделируемой системы. Динамика
таких процессов в первом приближении может быть описана динамическими
системами  с разрывными траекториями и управлениями импульсного типа,
т.е. такими, при которых происходит резкое, скачкообразное изменение
состояния системы в отдельные моменты времени. 
