\section*{Введение}
\addcontentsline{toc}{section}{Введение}
\label{intro}

Существует ряд прикладных задач, связанных с динамическими
управляемыми системами, когда необходимо понять как будет вести себя
система при различных управлениях. Для этого необходимо понять, как
устроено множество достижимости системы, что даст наиболее полную
информацию о возможностях системы. Например оценка множества
достижимости позволяет понять, как ведет себя система при неизвестных
внешних воздействиях, скажем воздействие ветра на самолет во время 
посадки, и позволит определить --- не выходит ли его параметры за грань 
допустимых. Кроме того знание множества достижимости может помочь находить 
оптимальные управления для различных функционалов.

В управляемых системах возможны ситуации, когда за
очень короткий период времени происходит существенное изменение
некоторых параметров, описывающих функционирование системы.  Важные
примеры подобных ситуаций можно найти в математической экологии,
например, рациональное использование и охрана водных, земельных,
атмосферных, минеральных и энергетических ресурсов, аварийный разлив
нефти и моделирование распространения нефтяного пятна,  а также в
экономике, механике, ракетодинамике,  квантовой электронике (лазерное
излучение является импульсным по своей природе), робототехнике,
медиотерапии  и т.д. В частности, такие ситуации возникают при
моделировании реальных процессов, управление которыми осуществляется в
течение столь кратковременных промежутков, что их можно принимать как
мгновенные, а результаты воздействия приводят к быстрому изменению
процесса --- скачкам фазовой траектории моделируемой системы. Динамика
таких процессов в первом приближении может быть описана динамическими
системами  с разрывными траекториями и управлениями импульсного типа,
т.е. такими, при которых происходит резкое, скачкообразное изменение
состояния системы в отдельные моменты времени. 

Основная цель дипломной работы состояла в изучении возможности и
попытке реализовать алгоритм численного построения множества
достижимости импульсной управляемой системы билинейной
структуры. Наиболее важным реультатом стала реализация идеи
динамического программирования для алгоритма, что позволило в разы
увеличить скорость, с которой строится аппроксимация системы.


%%% Local Variables:
%%% mode: latex
%%% TeX-master: "rs-ids"
%%% End:
