\section*{Введение}
\addcontentsline{toc}{section}{Введение}
\label{intro}

Многие теоретические и прикладные задачи сводятся к предельно упрощенной модели распространения волн, 
в которой отсутствуют формы колебаний, и для каждой точки области вычисляется лишь первый момент прохождения 
фронта волны через точку. Подобное упрощение выглядит логично, например, при моделировании фронта лесного пожара. 
Также оно широко применяется в геометрической оптике и в моделях распространения сейсмических волн. 
Математическая формализация задачи носит название уравнения эйконала и представляет собой нелинейное уравнение 
в частных производных первого порядка
$$|\nabla T| = \frac{1}{F(x)}, \quad x \in \Omega \subset \mathbb{R}^n$$
где неизвестная функция $T=T(x)$ --- время, за которое волна достигает точки $x$; $F(x)$ --- скорость прохождения сигнала,
зависящая от точки. Начальное условие задано на границе:
$$ T(x) = 0, \quad x \in \partial \Omega. $$
Иногда рассматриваются задачи с  начальным условием, заданным в одной точке.

Функцию $T(x)$ можно понимать как длину кратчайшего пути от множества $\partial \Omega$ до точки $x$ в метрике, 
определяемой функцией $F$, или кратчайшего времени, которое можно получить как решение задачи быстродействия, если $F(x)$ 
--- это скорость движения в точке $x$.

Из постановки задачи видно, что она не всегда имеет решение в классическом смысле. Во-первых, определенное таким образом решение, 
исходя из физического смысла, как правило, не будет всюду дифференцируемым. Во-вторых, функция, удовлетворяющая 
уравнению почти всюду, не единственна. Более того, таких функций бесконечно много. К счастью, существует несколько способов математически 
строго определить именно то решение, которое соответствует минимальному времени прихода сигнала. Это решение называется минимаксным или 
вязким (вязкостным).

Поиск кратчайшего расстояния на некотором множестве после дискретизации (введения некоторой сетки) сводится к разностной схеме. 
Если под расстоянием понимать  $\l_1$-метрику (манхеттенскую), то можно применять к прямоугольной сетке классические алгоритмы 
поиска кратчайших путей на графе (например, алгоритм Дейкстры). Евклидову расстоянию соответствует разностная схема с уравнениями 
второй степени, причем квадратное уравнение решается только тогда, когда известны значения $T$ в двух и более соседних узлах.
Вариантом алгоритма Дейкстры, применимым в данном случае, является так называемый Fast Marching Method (FMM), впервые предложенный и обоснованный 
 в работах J. Sethian и впоследствии получивший множество вариантов и обобщений. Другой широко используемый алгоритм --- Fast Sweeping Method (FSM) 
 --- работает с тем же набором разностных уравнений, но обновляет значения $T$ в узлах, чередуя несколько заранее выбранных порядков рассмотрения узлов. 
 Характерным и наиболее полезным свойством данных методов является время работы не хуже $\Theta(N \log N)$, где $N$ --- число узлов сетки.
 
 Вопрос о сходимости данных методов следует разделять на две части: 1) сходимость разностной схемы 2) сходимость конкретного алгоритма решения системы разностных уравнений. 
 Если первое выполняется для любого уравнения эйконала, в том числе с зависимостью скорости распространения сигнала от направления (анизотропией), определенной через гладкие функции, 
 то второе в анизотропном случае требует дополнительных предположений относительно взаимного расположения сетки, градиента решения и характеристик уравнения (кривых, вдоль которых распространяется информация). Для классического (изотропного) уравнения эйконала сходимость есть всегда.
 Отсутствие сходимости FSM или FMM означает, что решение системы разностных уравнений потребует большего времени, чем $\Theta(N \log N)$.
 
Оба этих алгоритма имеют варианты, предназначенные для решения уравнений более общего вида --- некоторого класса стационарных уравнений Гамильтона-Якоби, 
к которым сводятся, в частности, задачи планирования маршрутов в робототехнике и задачи быстродействия из теории оптимального управления.
Существуют варианты алгоритмов, работающие на неравномерных сетках, а также на триангуляциях. Также существуют параллельные реализации, хотя хорошую масштабируемость 
при большом числе потоков и распределенной памяти получить весьма трудно.

Целью данной работы является разработка реализаций  FMM и FSM на языке C и применение их к следующим задачам: 1) аппроксимации множества достижимости импульсной системы, 
2) вычисления расстояний в некоторой заданной метрике, 3) восстановления формы тела по черно-белому снимку. Новизна заключается в решении первой из этих задач, поскольку ранее методы данного типа для ее решения не применялись. Кроме того, была оценена целесообразность параллельного решения данных задач.

%%% Local Variables:
%%% mode: latex
%%% TeX-master: "eikonal_solver"
%%% End:
