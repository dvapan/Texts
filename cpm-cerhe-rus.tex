\documentclass[a4paper,12pt]{article} 
% отделять первую строку раздела абзацным отступом тоже
\usepackage{indentfirst} 

\usepackage{geometry} 
\geometry{left=1.5cm} 
\geometry{right=1.5cm}
\geometry{top=2.5cm} 
\geometry{bottom=1.5cm}

\usepackage[unicode=true]{hyperref}
\usepackage[warn]{mathtext}
\usepackage[utf8]{inputenc} 
\usepackage[english,russian]{babel}
\usepackage{indentfirst} 

\usepackage{amsmath}

\usepackage{graphicx} 
\usepackage{setspace}

\usepackage[labelsep=period]{caption}

\title{Эффективный метод динамических расчетов элементов
теплоэнергетических установок, сводящий решение систем дифференциальных
уравнений в частных производных к решению задач линейного программирования}

\author{Данил Апанович}

\begin{document}

\maketitle

\section{Введение} 

Расчеты стационарных и нестационарных режимов работы ряда элементов
теплоэнергетических установок (теплообменников различных типов, топок,
камер сгорания, турбинных ступеней и др.) сводятся к решению систем
дифференциальных уравнений в частных производных (СДУЧП). Основными
методами решения таких систем являются методы конечных разностей (МКР),
методы контрольных объемов (МКО) и методы конечных элементов (МКЭ).  

При использовании МКР на расчетной области строится сетка и для каждого ее
узла, на основе исходных дифференциальных уравнений, формируется подсистема
алгебраических уравнений [1–4]. В этих уравнениях частные производные
заменяются соответствующими конечными разностями. Подсистемы алгебраических
уравнений отдельных узлов сетки объединяются в единую систему
алгебраических уравнений, к которой добавляются краевые условия. Следует
отметить, что при этом точность решения СДУЧП сильно зависит от величин
шагов сетки по пространственным координатам и по времени. Стремление
поднять точность решения приводит к сокращению размеров шагов и
соответственно к увеличению числа узлов сетки и размерности системы
алгебраических уравнений. Во многих случаях эта размерность становится
столь большой, что система не может быть решена как единое целое без
использования тех или иных методов декомпозиции.  Снижения размерности
системы можно добиться использованием сетки с переменными шагами, но это
сильно усложняет алгоритм решения задачи, что особенно ощутимо для
расчетного пространства со сложной геометрией.

МКО применим к задачам, в которых дифференциальные уравнения отражают
законы сохранения массы (полной или отдельных химических элементов),
энергии и импульса [3, 4]). К таким задачам относится большинство задач
тепломассообмена.  Поэтому данный метод наиболее широко используется в
вычислительной гидрогазодинамике. В соответствии с МКО расчетная область
разбивается на контрольные объемы, для которых допустима неправильная
геометрическая форма.  Для каждого объема формируются балансовые уравнения,
учитывающие обмен данного объема с соседними объемами массой, энергией и
импульсом. Эти уравнения являются алгебраическими, в которых производные
заменяются на конечные разности, определяемые по значениям соответствующих
параметров в геометрических центрах смежных контрольных объемов. Кроме того
в уравнения входят площади граничных поверхностей между смежными
контрольными объемами. Причем балансы массы, энергии и импульса соблюдаются
для контрольных объемов вне зависимости от места расположения разделяющих
смежные объемы поверхностей. МКО позволяют более точно и более просто чем
МКР представить сложную расчетную область.  К недостаткам как МКР, так и
МКО следует отнести невозможность расчета искомых переменных в точках, не
являющихся узлами сетки или центрами контрольных объемов.

МКЭ первоначально предназначались для статических расчетов строительных
конструкций [7–9]. Они основаны на разбиении расчетной области на
достаточно большое число конечных элементов простой формы, как правило,
многогранников.  На каждом элементе выделяются узлы. В первую очередь это
вершины многогранников, однако, возможен выбор в качестве узлов и других
точек. Для каждого элемента для всех искомых из системы дифференциальных
уравнений функций ищутся линейные комбинации заранее заданных базисных
функций, связывающие пространственные координаты и время с соответствующей
искомой переменной. Совокупность таких комбинаций для всех элементов должна
отвечать следующим условиям: достигается минимум суммы квадратов невязок
для всех узлов всех конечных элементов (невязки получаются при подстановке
в дифференциальные уравнения нужных производных соответствующих линейных
комбинаций базисных функций); равенство искомых переменных в вершинах
смежных элементов при определении их из линейных комбинаций базисных
функций этих элементов; равенство расчетных краевых условий при определении
их на основе соответствующих линейных комбинаций базисных функций. Следует
отметить, что при согласованном подборе числа конечных элементов, числа
узлов в элементах и числа базисных функций можно добиться того, что невязки
в узлах элементов при соблюдении указанных условий окажутся равными нулю.
При этом количество невязок должно быть равно количеству линейных
размерностей базисных функций.  Указанные условия порождают систему
алгебраических уравнений, решение которой дает линейные комбинации базисных
функций, позволяющие определить искомые переменные в любой точке расчетной
области, что является несомненным достоинством МКЭ. Следует отметить, что
если исходная СДУЧП линейная, то и системы алгебраических уравнений, к
которым сводится приближенное решение СДУЧП, будут линейным.

При решении иных нестационарных задач с использованием МКР, МКО и МКЭ
получающиеся системы алгебраических уравнений становятся чрезвычайно
большими и для их решения используются методы декомпозиции, состоящие, как
правило, в разделении решения по пространственным координатам и по времени.
Выделяется подсистема уравнений, относящаяся к одному моменту времени.
После ее решения находятся частные производные искомых величин по времени.
С использованием этих производных определяются значения соответствующих
величин в следующий момент времени (на следующем временном слое). При этом
используются различные явные и неявные разностные схемы [10].

В рассмотренных методах условием малого отклонения приближенного решения
СДУЧП от ее точного решения является малость величин характерных
геометрических размеров (шаги сетки, максимальные размеры контрольных
объемов и конечных элементов). Наиболее обоснованным численным критерием
такого отклонения (качества приближенного решения) является значение
максимальной по модулю невязки во всех рассматриваемых (контрольных) точках
расчетной области. Однако ни в одном из рассмотренных методов данный
критерий не используется. 

С учетом указанных недостатков МКР, МКО и МКЭ предлагается более
эффективный метод решения СДУЧП. Он основан на поиске таких значений
коэффициентов линейных разложений базисных функций, представляющих
зависимости искомых из СДУЧП функций от пространственных координат и
времени, при которых минимального значения достигает максимальная по модулю
невязка, определяемая среди всех невязок в заданных контрольных точках
расчетной области. Переход от минимизации суммы квадратов невязок к
минимизации максимальной по модулю невязки значительно улучшает качество
приближенного решения и позволяет перейти от малых конечных элементов к
достаточно крупным блокам, в пределах каждого из которых ищутся свои
линейные разложения базисных функций. В основе метода лежит назначение в
пределах расчетной области контрольных точек, в каждой из которых
определяются невязки. Важно подчеркнуть, что количество невязок должно быть
больше, причем существенно, числа коэффициентов линейных разложений
базисных функций.

Все контрольные точки расчетной области делятся на три группы. Первая
группа — это внутренние контрольные точки блоков. В данных точках
рассчитываются только невязки исходных дифференциальных уравнений,
получающиеся после подстановки в них искомых функций и их частных
производных, определяемых из линейных разложений базисных функций. Вторая
группа — это точки, лежащие на границах блоков. В этих точках невязки
дифференциальных уравнений рассчитываются для каждого смежного блока с
использованием его линейных разложений. Кроме того определяются невязки
между искомыми функциями смежных блоков. К третьей группе относятся
контрольные точки, лежащие на границах расчетной области. В этих точках
состав невязок дифференциальных уравнений дополняется невязками,
определяющими точность приближения полученного решения к начальным и
граничным условиям. В частности, определяются невязки между заданными
значениями величин на границах расчетной области и рассчитанными из
линейных комбинаций базисных функций значениями этих величин.

Если расчетная область делится на подобласти, каждая из которых описывается
своей системой дифференциальных уравнений, то на блоки делятся указанные
подобласти. В точках, лежащих на границах смежных подобластей определяются
невязки между значениями тех искомых функций, которые входят в системы
дифференциальных уравнений обеих подобластей.

Следует отметить, что при минимизации максимальной по модулю невязки
приходится сравнивать невязки, имеющие различную размерность и различный
физический смысл.  Поэтому такое сравнение возможно проводить лишь между
относительными невязками, получающимися при делении абсолютных невязок на
их максимально допустимые значения.

Если исходная СДУЧП является линейной, то предлагаемый метод, который можно
назвать методом контрольных точек (МКР), сводится к решению задачи
линейного программирования [11–12].

\section{Математическая постановка задачи}

Данная постановка сформирована с учетом особенностей динамических расчетов
элементов теплоэнергетических установок, в частности таких сложных как
высокотемпературные керамические теплообменники периодического действия.

Имеется расчетная область $Q$. Каждая точка этой области определяется
$N$-мерным вектором $X$. В пространстве $N$ параметров, определяющих
область $Q$ протекает рассчитываемый динамический процесс. Как правило
данные параметры это пространственные координаты и время.

В некоторых случаях динамические процессы, протекающие в различных частях
рассчетной области $Q$ могут описываться различными системами
дифференциальных уравнений. В связи с этим полагаем, что область $Q$
делится на $L$ подобластей, в каждой из которых действует своя система
дифференциальных уравнений в частных производных. Принимается, что для
подобласти $Q_l$ система включает $K_l$ дифференциальных уравнений.
Искомыми для этой подобласти являются $K_l$ функций вида

\begin{equation} 
    \begin{array}{ll} 
        y^l_1=f^l_1(x), \\ 
        y^l_2=f^l_2(x), \\
        \cdots \\ 
        y^l_{K_l}=f^l_{K_l}(x).  
    \end{array} 
    \label{eq:desir-fnc}
\end{equation}

При этом $k$-ое дифференциальное уравнение в общем виде зависит от функций:
$y^l_i, i = 1,\ldots,K_l$, их первых производных $(y^l_{ij})', i =
1,\ldots,K_l, j = 1, \ldots, N$ (общее число первых производных: $K_l \cdot
N$), их вторых производных $(y^l_{ijq})'', i = 1,\ldots,K_l, j =
1,\ldots,N, q = 1,\ldots,N$ (с учетом того, что $(y^l_{ijq})'' =
(y^l_{iqj})''$ общее число вторых производных $K_l \cdot (N^2+N) / 2$) и
т.д., где $i$ --- номер функции, $j$, $q$ --- номера переменных по которым
осуществляется дифференцирование. Отметим, что в отдельные дифференциальные
уравнения $l$-ой подобласти могут входить не все искомые функции $y^l_i$ и
все их производные.  Однако в дальнейшем для простоты описания задачи будем
считать, что в каждое дифференциальное уравнение входят все функции
$y^l_1,\ldots,y^l_{K_l}$, первые производные всех функций по всем
переменным $x$, $(y^l_{11})', \ldots (y^l_{11})'$, и все возможные значения
вторых производных $(y^l_{111})'', \ldots, (y^l_{K_lNN})''$ и т.д. Тогда
систему дифференциальных уравнений подобласти $Q_l$ можно представить в 
виде

\begin{equation}
    D^{lj}(y^l_1,\ldots,y^l_{K_l}, (y^l_{11})',\ldots, (y^l_{K_l,N})',
           (y^l_{111})',\ldots, (y^l_{K_lNN})'',\ldots) = 0,
    \label{eq:pde}
\end{equation}
$j = 1, \ldots, K_l$, где $j$ --- номер дифференциального уравнения в 
системе дифференциальных уравнений $Q_l$.

Отметим, что среди подобластей $Q_l$, $l=1,\ldots,K_l$ можно видеть пары
смежных подобластей. Если некоторая точка $x$ принадлежит границе пары
смежных подобластей $Q_i$, и $Q_j$ (обозначим эту границу через
$\Gamma_{ij}$), то некоторые функции искомые из системы уравнений
подобластей $Q_i$ равны в этой точке соответсвующим (т.е. имеющим тот же
физический смысл) функциям определяемым из системы уравнений подобласти
$Q_j$. Отметим, что в общем случае номер некоторой функции в системе
уравнений подобласти $Q_i$ может не совпадать с номером соответсвующей ей
функции в системе уравнений подобласти $Q_j$. Если подобласти $Q_i$ и $Q_j$
не являются смежными (т.е. не имеют общих точек), то $\Gamma_{ij} =
\emptyset$. Для каждой пары подобластей $Q_i$ и $Q_j$ вводится множество
пар номеров соответсвующих функций.

\begin{equation}
    S^{ij}=\left\{(n^{ij}_1,p^{ij}_1),(n^{ij}_2,p^{ij}_2),
    \ldots,(n^{ij}_{K^{соот}_{ij}},p^{ij}_{K^{соот}_{ij}})\right\},
    \label{eq:set-nums}
\end{equation}
где $n^{ij}_k$ --- номер функции в системе уравнений подобласти $Q_i$
в $k$-ой паре соответсвия, входящей в $S^{ij}$, $p^{ij}_k$ --- номер 
функции в системе уравнений подобластей $Q_j$ в $k$-ой паре соответствия,
входящей в $S^{ij}$, $K^{соот}_{ij}$ --- число пар соответствия для 
систем уравнений подобластей $Q_i$ и $Q_j$. Если $S^{ij} = \emptyset$, то
система уравнений подобластей $Q_i$ и $Q_j$ не являются смежными. Если
$\bar{x} \in \Gamma^{ij}$, то должны выполнятся условия

\begin{equation}
    \begin{array}{ll}
        y^i_{n^{ij}_1}(\bar{x})=y^j_{p^{ij}_1}(\bar{x}),\\
        y^i_{n^{ij}_2}(\bar{x})=y^j_{p^{ij}_2}(\bar{x}),\\
        \cdots \\
        y^i_{n^{ij}_{K^{соот}_{ij}}}(\bar{x})=
        y^j_{p^{ij}_{K^{соот}_{ij}}}(\bar{x}).\\
    \end{array}
    \label{eq:bnd-cnd-1}
\end{equation}

Для каждой подобласти $Q_l$ вводятся множества внешних границ 
$\Gamma^{li}_{внеш}$, $i=1,\ldots,I^{l}_{внеш}$, где $I^l_{внеш}$ --- число
внешних границ подобласти $Q_l$. Для каждой границы $\Gamma^{li}_{внеш}$ 
вводится множество номеров искомых функций, значения которых на данной
границе задано $$J^{li}_{внеш}=\left\{j^{li}_{внеш 1},\ldots,
j^{li}_{внеш K^{li}_{внеш}}\right\},$$ где $K^{li}_{внеш}$ --- число 
функций, задаваемое на границе $\Gamma^{li}_{внеш}$. В каждой точке
$\bar{x} \in \Gamma^{ij}_{внеш}$ должны выполнятся условия

\begin{equation}
    \begin{array}{ll}
        y^l_{j^{li}_{внеш }1}(\bar{x})=
          y^{lz}_{j^{li}_{внеш }1}(\bar{x}),\\
        y^l_{j^{li}_{внеш }2}(\bar{x})=
          y^{lz}_{j^{li}_{внеш }2}(\bar{x}),\\
        \cdots \\
        y^l_{j^{li}_{внеш} K^{li}_{внеш}}(\bar{x})=
        y^{lz}_{j^{li}_{внеш} K^{li}_{внеш}}(\bar{x}).\\
    \end{array}
    \label{eq:bnd-cnd-2}
\end{equation}
Надстрочным индексом $z$ обозначаются заданные функции от независимых
параметров.

Как видно, если функции $y^l_1=f^l_1(x),\ldots,y^l_{K_l}=f^l_{K_l}(x)$,
$l=1,\ldots,L$ являются решениями систем дифференциальных уравнений в
частных производных описывающих подобласти $Q_1,\ldots,Q_l$, то для
каждой точки $x \in Q_l$ должны выполняться условия \eqref{eq:pde} для
каждой точки $\bar{x} \in \Gamma^{ij}$ если $\Gamma^{ij} \in \emptyset$
должны выполняться краевые условия \eqref{eq:bnd-cnd-1}, а для каждой 
внешней границы множества $Q_l$ должны выполняться краевые условия
\eqref{eq:bnd-cnd-2}. 

В настоящей работе вместо неизвестных функций вида \eqref{eq:desir-fnc}
являющихся точными решениями, отвечающх условиям \eqref{eq:pde},
\eqref{eq:bnd-cnd-1},\eqref{eq:bnd-cnd-2} ищется приближенное решение
в виде линейных комбинаций базисных функций. В качестве базисных функций
используются функции вида $x^{i_1}_1 \cdot x^{i_2}_2 \cdot \ldots \cdot
x^{i_N}_N$, где $x_l$ --- $l$-ая комбинация вектора $x$, $i_1,i_2,\ldots
i_N$ - показатели степени, отвечающие условиям

\begin{equation}
    0 \le i_j \le S,
    \label{powcnd1}
\end{equation}

\begin{equation}
    \sum^{N}_{i=1} i_j \le S.
    \label{powcnd2}
\end{equation}
Искомые функции представляются в виде

\begin{equation}
    f(x_1,\ldots,x_N)=\sum^{c}_{k=1} a_kx^{i_{1k}}_1 
    \cdot\ldots\cdot x^{i_{Nk}}_N,
\end{equation}
где $C=N^{S+1}$ --- число всеъ возможны сочетаний степенй, отвечающих 
условиям \eqref{powcnd1}, \eqref{powcnd2}, $i_{jk}$~---~показатель степени
в $k$-ом элементе сочетания при $j$-ой компоненте вектора $X$. Коэффициенты
$a_k$ являются искомыми коэффициентами линейного разложения базисных 
функций.


\end{document}
