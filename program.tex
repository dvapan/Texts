\section{Программа}
\label{sec:program}

При реализации алгоритма использовался язык программирования python
вместе с библеотекой для научных и инженерных расчетов scipy.

Основу кода состовляет класс \emph{ImpulseSystem}, который хранит
функции $f(x)$ и $G(x)$, позволяя вычислять множество достижимости для
различных начальных данных и параметров времени. 

\begin{lstlisting}[language=Python,
caption={Интерфейс Impulse System}]
class ImpulseSystem(object):
    drift = None
    impulse = None

    def __init__(self, drift=None, impulse=None):
        self.drift = drift
        self.impulse = impulse
        self.x_0 = None
        self.ht = None
        self.h = None
        self.time = None
        self.value = None
        self.cost = dict()
        self.state_space = dict()
        self.grid_error = 0

    def integrate(self, x_0, ht, h, time, value):

    def impulse_jump(self, points_queue):

    def make_drift(self):

    def get_costs_at(self, i, j):

    def get_cost(self, cur_matrix, x, y):
\end{lstlisting}

Принци работы весьма прост. Мы создаем экземпляр класса
\emph{ImpulseSystem} отдавая ему в конструктор функции отвечающие за
дрейф --- \emph{drift} и импульсное воздействие ---
\emph{impulse}. Например для первого примера \ref{sec:snwd}. Созданный объект будет
выглядеть следующим образом:

\begin{lstlisting}[language=Python,
caption={Интерфейс Impulse System}]
F = None
G = lambda x: sc.matrix([[1 - x[1], 0], [0, 1 - x[0]]])
isys = ImpulseSystem(drift=F, impulse=G)
\end{lstlisting}

Для запуска процесса интегрирования мы вызываем метод
\emph{integrate}. На вход которому подаем все параметры --- точку
$x_0$, шаг по времени $h_t$, шаг по сетке $h$, время $T$ и ресурс $V$.

Общий процесс идет по принципу, что описан выше смотри
\ref{sec:wave_alg}. На каждом интервале сначала работает метод
\emph{impulse jump}, затем все полученные точки дрейфуют под действием
\emph{make drift} и так для всех интервалов.

После завершения работы программы мы получаем список точек, для
аппрокисмации множества достижимости.



%%% Local Variables:
%%% mode: latex
%%% TeX-master: "rs-ids"
%%% End:
