\chapter{Алгоритмы аппроксимации}
\label{ch:algrhtms}


В этой главе опишем два подхода для численной аппроксимации множества
достижимости. Первый подход, основывался на идее полного перебора всех
возможных управлений, что не позволяло хоть какое-то его адекватное
практическое применение для решения задач.  Второй подход, основанный
на численном решении волнового уравнения Эйконала и его связи с
уравнением Гамильтона-Якоби, позволил использование быстрых алгоритмов
поиска кратчайшего пути на сетке. Появился в результате серьезного
переосмысления первого.

\section{Переборный алгоритм}
\label{sec:simple_alg}

Отметим, что интегральное ограничение в
\eqref{system_s} и условия, накладываемые на функции $f(t,x)$ и $G(t,x),$
обеспечивают компактность и связность множества достижимости, а
также его непрерывную зависимость от величины интегрального
ресурса управления $V$. Проведем преобразование системы
\eqref{system_s} при помощи разрывной замены времени. 

Аппроксимация множества достижимости импульсной системы осуществляется по
следующей схеме:
\begin{enumerate}
\item Задается разбиение отрезка $[b-a,b-a+V]$ на $k$ частей точками:
  $ \Delta=\big\{ \tau_{0}, \tau_{1}, \ldots, \tau_{k} \big\}, $ где
  $\tau_{0}=b-a< \tau_{1}< \ldots < \tau_{k}=b-a+V$. Положим
  $h_i:=\tau_{i}-\tau_{i-1},$ $i=\overline{1,k}$;
\item для каждого фиксированного $\tau_{i},$ $i=\overline{1,k},$
  численно строится оценка множества достижимости вспомогательной
  системы $\mbox{\bf R}_i ,$ т.е.
  ${\mathcal R}_{\mbox{\rm auxiliary}}(\tau_{i}) \approx \mbox{\bf
    R}_i$;
\item оценку множетсва достижимости исходной импульсной системы \eqref{system_s} с
  разрывными траекториями получаем путем объединения множеств
  $\mbox{\bf R}_i$, $i=\overline{0,k},$ т.е.
  $ {\mathcal R}_M(b)\approx \bigcup_{i=0}^k \mbox{\bf R}_i$.
\end{enumerate}


Рассмотрим кусочно-постоянные управления
$(\omega_0(\cdot),\omega_1(\cdot),\omega_2(\cdot))$, соответствующие
$\Delta$ и удовлетворяющие условиям:
\begin{equation*}
  \begin{array}{l}
    \sum_{j=1}^k \omega_{0j}h_j=b-a\\
    \omega_{0j}, \omega_{1j}, \omega_{2j} \geq 0,\\ 
    \omega_{0j}+ \omega_{1j}+ \omega_{2j} =1, \ \ j=\overline{1,k}.
  \end{array}
\end{equation*}

Пусть $\Omega(\Delta)$ --- множество всех таких управлений, компоненты
которых принимают значения либо 0 либо 1. Тогда всевозможные выпуклые
комбинации элементов $\Omega$ дают все кусочно постоянные управления,
соответствующие выбранному разбиению.  Заметим, что количество
элементов в $\Omega(\Delta)$ не превышает $m^{k-q} C_k^q$, где
$q=\left[\frac{b-a}{h}\right]$ --- число отрезков, на которых
$\omega_{0j}=1$.  Элементы $\Omega(\Delta)$ полностью определяются
номерами компонент с единичными значениями. Например, если $T=[0,1]$,
$\tau_1=3$ --- конечный момент времени во вспомогательной системе,
$k=3$ --- количество подотрезков разбиения, то $\Omega(\Delta)$
задается множеством
$\left\{122; 123 ; 132 ; 133 ;212 ; 213; 312 ; 313 ; 221; 231 ; 321 ;
  331 \right\}$.
Затем мы строим точки множества достижимости для каждого управления из
$\Omega(\Delta)$, и, если нужно, их попарные выпуклые комбинации.

Алгоритм такого имеет очень серьезный недостаток --- экспоненциальную
вычсилительную сложность, как это показано выше в оценук мощности
множества $\Omega(\Delta)$. Данный недостаток, накладывает серьезные
ограничения на практическую применимость данного алгоритма.

\section{Волновой алгоритм}
\label{sec:wave_alg}
Введем равномерную сетку на области измененияи на отрезке времени и
разделим два способа изменения состояния системы — дрейф системы и
движение под воздействием импульса.  Для всех узлов можно посчитать
стоймость перехода из него в соседние, получая при этом взвешенный
граф, для импульсного воздействия.

Стартовому узлу присваиваем максимальный ресурс системы \eqref{system_s}. Затем
подсчитываем значение ресурса для соседних узлов, после этого соседние
для них и т.д пока не значение ресурса в узлах не станет нулевым. При
этом, если в один и тот же узел приходят разные значения ресурса,
записывается большее — таким образом мы найдем насколько далеко можно
добраться из стартовой точки, т.е. границу МД системы. \eqref{system_s}. Таким
образом реализуется идея поиска кратчайшего пути на графе в смысле
оптимального расходования ресурса управления системы.

Далее мы действуем дрейфом на все полученные точки, в течении одного
малого шага отрезка , смещая тем самым узлы. После чего попытаемся из
кадого узла подействовать импульсом (если это позволяет попасть в
соседний узел с большим значением .  Таким образом действуя по
переменке дрейфом и импульсным скачком мы пройдем весь отрезок времени
и получим точки МД для системы.
