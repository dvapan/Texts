\section{Алгоритмы аппроксимации}
\label{sec:algrhtms}


В этой главе опишем два подхода для численной аппроксимации множества
достижимости. Первый подход, основывался на идее полного перебора всех
возможных управлений, что не позволяло хоть какое-то его адекватное
практическое применение для решения задач.  Второй подход, основанный
на численном решении волнового уравнения Эйконала и его связи с
уравнением Гамильтона-Якоби, позволил использование быстрых алгоритмов
поиска кратчайшего пути на сетке. Появился в результате переосмысления
первого.

\subsection{Переборный алгоритм}
\label{sec:simple_alg}

Отметим, что интегральное ограничение в
\eqref{system_s} и условия, накладываемые на функции $f(t,x)$ и $G(t,x),$
обеспечивают компактность и связность множества достижимости, а
также его непрерывную зависимость от величины интегрального
ресурса управления $V$. Проведем преобразование системы
\eqref{system_s} при помощи разрывной замены времени. 

Аппроксимация множества достижимости импульсной системы осуществляется по
следующей схеме:
\begin{enumerate}
\item Задается разбиение отрезка $[b-a,b-a+V]$ на $k$ частей точками:
  $ \Delta=\big\{ \tau_{0}, \tau_{1}, \ldots, \tau_{k} \big\}, $ где
  $\tau_{0}=b-a< \tau_{1}< \ldots < \tau_{k}=b-a+V$. Положим
  $h_i:=\tau_{i}-\tau_{i-1},$ $i=\overline{1,k}$;
\item для каждого фиксированного $\tau_{i},$ $i=\overline{1,k},$
  численно строится оценка множества достижимости вспомогательной
  системы $\mbox{\bf R}_i ,$ т.е.
  ${\mathcal R}_{\mbox{\rm auxiliary}}(\tau_{i}) \approx \mbox{\bf
    R}_i$;
\item оценку множетсва достижимости исходной импульсной системы \eqref{system_s} с
  разрывными траекториями получаем путем объединения множеств
  $\mbox{\bf R}_i$, $i=\overline{0,k},$ т.е.
  $ {\mathcal R}_M(b)\approx \bigcup_{i=0}^k \mbox{\bf R}_i$.
\end{enumerate}


Рассмотрим кусочно-постоянные управления
$(\omega_0(\cdot),\omega_1(\cdot),\omega_2(\cdot))$, соответствующие
$\Delta$ и удовлетворяющие условиям:
\begin{equation*}
  \begin{array}{l}
    \sum_{j=1}^k \omega_{0j}h_j=b-a\\
    \omega_{0j}, \omega_{1j}, \omega_{2j} \geq 0,\\ 
    \omega_{0j}+ \omega_{1j}+ \omega_{2j} =1, \ \ j=\overline{1,k}.
  \end{array}
\end{equation*}

Пусть $\Omega(\Delta)$ --- множество всех таких управлений, компоненты
которых принимают значения либо 0 либо 1. Тогда всевозможные выпуклые
комбинации элементов $\Omega$ дают все кусочно постоянные управления,
соответствующие выбранному разбиению.  Заметим, что количество
элементов в $\Omega(\Delta)$ не превышает $m^{k-q} C_k^q$, где
$q=\left[\frac{b-a}{h}\right]$ --- число отрезков, на которых
$\omega_{0j}=1$.  Элементы $\Omega(\Delta)$ полностью определяются
номерами компонент с единичными значениями. Например, если $T=[0,1]$,
$\tau_1=3$ --- конечный момент времени во вспомогательной системе,
$k=3$ --- количество подотрезков разбиения, то $\Omega(\Delta)$
задается множеством
$\left\{122; 123 ; 132 ; 133 ;212 ; 213; 312 ; 313 ; 221; 231 ; 321 ;
  331 \right\}$.
Затем мы строим точки множества достижимости для каждого управления из
$\Omega(\Delta)$, и, если нужно, их попарные выпуклые комбинации.

Если не вдаваться в подробности, то алгоритм вычисления множества
достижимости будет состоять из следующих шагов
\begin{itemize}
\item[\bf Шаг 0.] Задаются функции и параметры управляемой системы: $f(x)$, $G(x)$, $T$, $V$, $x_0.$
\item[\bf Шаг 1.] Осуществляется преобразование к вспомогательной задаче
\item[\bf Шаг 2.] Задается разбиение отрезка $[0,t_1+V]$
  на $k$ частей:    \begin{equation} \Delta:=\big\{ \tau_{0}, \tau_{1}, \ldots,
    \tau_{k}  \big\},
  \end{equation} где $\tau_{0}=0 \leq \tau_{1}\leq \ldots \leq \tau_{k}=t_1+V.$

\item[\bf Шаг 3.] Для каждого $\tau_{i}\in \Delta$ задается управление $\omega_0, \omega$, так, что если $\omega_0 = 1$, то $\omega = (0,..,0)$, в противном случае $w = (0,...,0,1,0,..,0)$. При этом нужно учитывать, что $\sum\limits_{\forall \tau_{i}} \omega_{0_{\tau_{i}}} = t_1$
\item[\bf Шаг 4.] Интегрируем вспомогательную систему с полученным $\omega_0, \omega$, и получаем точку из множества достижимости
\end{itemize}


Алгоритм такого имеет очень серьезный недостаток --- экспоненциальную
вычсилительную сложность, как это показано выше в оценук мощности
множества $\Omega(\Delta)$. Данный недостаток, накладывает серьезные
ограничения на практическую применимость данного алгоритма.

\subsection{Пиксельный алгоритм}
\label{sec:wave_alg}

Основанием для алгоритма послужило численное решение уравнения типа
эйконала \cite{S1999} 

\begin{equation*}
  H(x,\nabla v_s(x)) = 0
\end{equation*}
для функции $v_s(x) = \inf \limits_{(z_s,\omega_s)} \int_0^\tau
| \omega_s(\xi) | d\xi$, где инфимум берется по всем траекториям
системы \eqref{system_d}, удовлетворящим концевым ограничениям $z_s(0)
= x_0, z_s(\tau) = x$

Введем равномерную сетку на области изменения $x=(x_1,x_2)$ с шагом
$h_x$ и на отрезке времени $T$ с шагом $h_t$.
Разделим два способа изменения состояния системы — дрейф системы, за
который отвечает $f(t,x)$ из \eqref{system_d} и
движение под воздействием импульса, за которое отвечает $G(x)$ из
\eqref{system_d}.

Для всех узлов можно посчитать стоймость перехода из него в соседние.
Ресурс $V$ расходуется на переход из каждой точки
$x_{ij} = (x_i,x_j)$, где $(x_i,x_j) = x_0 + (i,j)\cdot h_x$ в
соседние точкки. Эти точки имеют вид $x_{ij} +\Delta$, для которых
$\Delta$ принимает значения из множества --- $
\{(1,0),(-1,0),(0,1),(0,-1)\}$.
Благодаря тому, что $G(x)$ --- линейна по $x$, мы можем получить точку
$x_{ij} +\Delta \cdot h_x$. А если представить все это как шаг метода
Эйлера для решению дифференциальных уравнений, мы сможем получить
значение параметра $v$ для скачка:
\begin{eqnarray*}
  &x_{ij} +\Delta \cdot h_x = x_{ij} + G(x_{ij})\cdot v \Leftrightarrow\\
  &\Delta \cdot h_x = G(x_{ij})\cdot v \Leftrightarrow\\
  &v = G(x_{ij})^{-1} \Delta \cdot h_x
\end{eqnarray*}

Таким образом мы можем определить функцию $cost(i,j,\Delta)$ ---
стоимость перехода в соседнюю точку при действии импульса, как решение
системы линейных уравнений:
$$cost(i,j,\Delta) = G(x_{ij})^{-1} \Delta \cdot h_x$$ при этом
получим взвешенный граф с вершинами, являющимися узлами сетки и
ребрами ведущими в соседние узлы. Тогда траектория системы
\eqref{system_d} представляется путем на таком графе. В сочетании с
ресурсом $V$, смысл которого в том, что он тратится на каждом из ребер
графа в соотвествии с стоймостью этого ребра, мы можем представить
задачу численной аппрокисмации множества достижимости как задачу
поиска всех вершин графа в которые можно попасть из начальной вершины
(эквивалент $x_0$) с ресурсом $V$. 

Стартовому узлу присваиваем максимальный ресурс системы \eqref{system_s}. Затем
подсчитываем значение ресурса для соседних узлов, после этого соседние
для них и т.д пока не значение ресурса в узлах не станет нулевым. При
этом, если в один и тот же узел приходят разные значения ресурса,
записывается большее — таким образом мы найдем насколько далеко можно
добраться из стартовой точки, т.е. границу МД системы. \eqref{system_s}. Таким
образом реализуется идея поиска кратчайшего пути на графе в смысле
оптимального расходования ресурса управления системы.

Далее мы действуем дрейфом на все полученные точки, в течении одного
малого шага отрезка , смещая тем самым узлы. После чего попытаемся из
кадого узла подействовать импульсом (если это позволяет попасть в
соседний узел с большим значением.  Таким образом действуя по
переменке дрейфом и импульсным скачком мы пройдем весь отрезок времени
и получим точки МД для системы.

Таким образом вычисление можно свести к следующим шагам:
\begin{itemize}
\item[\bf Шаг 0.] Задаются функции и параметры управляемой системы:
  $f(x)$, $G(x)$, $T$, $V$, $x_0$, $h$, $ht$.
\item[\bf Шаг 1.] Вычисляются стоймости перехода между соседними
  узлами сетки, строится граф.
\item[\bf Шаг 2.] 
  Разбиваем отрезок времени $T$  на интервалы с шагом $ht$
\item[\bf Шаг 3.] 
  На первом интервале.
  Для каждого узла сетки находим наиболее короткий путь по графу,
  применяя идею динамического программирования: каждая следующая
  задача решается за счет предыдущих, решенных оптимально.
\item[\bf Шаг 4.] Для каждой точки считаем ее смещение под действием
  дрейфа, на отрезке шага $ht$
\item[\bf Шаг 5.] Шаги 3 и 4 повторяются для всех интервалов.
  
\end{itemize}

После прохождения по всему отрезку времени, мы и получаем конечное
состояние системы. Стоит отметить, что при небольшой модификации:
после каждого интервала интегрирования сохранять результат, мы получим
всю историю изменения системы в обычном времемени, что напоминает
дифференцирование обыкновенных дифференциальных уравнений.


%%% Local Variables:
%%% mode: latex
%%% TeX-master: "rs-ids"
%%% End:
