\chapter{Постановка задачи. Теория}
\label{ch:theory}

% Расскажем об импульсных динамических системах подробнее и подведем %
% соотв. теоретическую базу.
%\section {Импульсные управления}
\section {Управляемая система с разрывной траекторией}
\label{sec:csdisttrack}
% Опишем здесь множество траекторий импульсной
% управляемой системы с траекториями ограниченной вариации и ее
% множество достижимости в конечный момент времени.

Для начала взглянем на обычную управляемую систему, расширение
множества траекторий которой и приводит к импульсной системе. Будем
работать с системой следующего вида

\begin{equation}
  \label{system_s}
  \begin{array}{l}
    \dot{x}(t)=f\big(x(t)\big)+G\big(x(t)\big)v(t), \quad x(0)=x_0, \\[8pt]
    v(t)\geq 0  \qquad \forall\, t\in T = [0,t_1], \qquad
    \displaystyle\int_{0}^{t_1} ||v(t)||dt\leq V.
  \end{array} \tag{$S$}
\end{equation}

Здесь
\begin{itemize}
  \item $||v||:=\displaystyle\sum_{i=1}^m |v_i|$
  \item $x(\cdot)$ --- абсолютно непрерывная вектор-функция
    (траектория), $x(t)\in {\mathbb R}^n,$
  \item $v(\cdot)$ --- измеримая существенно ограниченная
    вектор-функция (управление),
  
  \item $V$ --- заданная величина интегрального ресурса управления
    $v$.
\end{itemize}

Пару функций $\bigl(x(\cdot),v(\cdot)\bigr),$ удовлетворяющих
системе \eqref{system_s}, будем называть процессом системы \eqref{system_s}.

Будем предполагать, что вектор-функция $f$ является функцией из
$L_{\infty}(T,\mathbb{R}^2)$, а $G$ задана следующим правилом:

\begin{equation*}
  G(x) = 
  \begin{pmatrix}
    a x_2+b & 0\\ 0 & c x_1 +d 
  \end{pmatrix}
\end{equation*}
     

Отметим особенность системы \eqref{system_s}. В ней управление $v$ входит в
правую часть системы линейно и может принимать сколь угодно большие
значения. Это позволяет траекториям системы \eqref{system_s} сколь угодно близко
(в поточечном смысле) приближаться к разрывным функциям.  Например,
последовательность $\{v_k(\cdot)\}$, состоящая из допустимых
$v$-управлений:

\begin{equation*} 
  v_k(t)=\left\{ 
    \begin{array}{ll}
      0, & t \in \left[0,\tau\right)\\
      kV,  & t\in \left[\tau,\tau+\frac{1}{k}\right)\\
      0,  & t\in \left[\tau+\frac{1}{k},t_1\right]
    \end{array}
  \right.
\end{equation*}

Такая последовательность
называется дельтообразной. В данном случае она моделирует в точке
$t=\tau$ импульс интенсивности $V$.  Положим $f\equiv 0,$
$G\equiv 1$. Тогда соответствующая последовательность траекторий:
\begin{equation*}
  x_{k}(t)=\left\{
    \begin{array}{ll}
      x_0, & t \in \left[0,\tau \right), \\
      x_0+kV(t-\tau) & t \in \left[\tau,\tau+\frac{1}{k} \right),\\
      x_0+V, & t\in [\tau,t_1].
    \end{array} \right.
\end{equation*}

поточечно сходится к разрывной функции

\begin{equation*}
  x_{\infty}(t)=\left\{
    \begin{array}{ll}
      x_0, & t \in \left[0,\tau \right), \\
      x_0+V, & t\in [\tau,t_1].
    \end{array} \right.
\end{equation*}

Данная особенность системы \eqref{system_s} приводит, в частности, к тому, что
поставленные в ней задачи оптимизации, как правило, не будут иметь
решения в классе обычных процессов, т.е. абсолютно непрерывных
траекторий и измеримых ограниченных управлений. При этом
минимизирующие последовательности траекторий могут поточечно сходиться
к разрывным функциям.

Множество обычных траекторий системы \eqref{system_s} будет состоять
из функций, имеющих равномерно ограниченные полные вариации на отрезке
$T$. Следовательно, любая последовательность траекторий будет
содержать подпоследовательность, поточечно сходящуюся к некоторой
функции ограниченной вариации. Именно такие функции, являющиеся
поточечными пределами последовательностей обычных траекторий, в
дальнейшем будут называться разрывными (или обобщенными) решениями
системы \eqref{system_s}.

В случае скалярного управления обобщенные траектории будут устойчивы.
Это значит, что при любой аппроксимации импульсного воздействия обычными
управлениями соответсвующие траектории системы сойдутся к одной и той
же разрывной траектории. Если же управление векторное наблюдается
феномен зависимости предельной разрывной траектории от способа
аппроксимации импульсного управления и возникает бесчисленное
множество разрывных траекторий. Обобщенные траектории будут устовйчивы
при выполнении условия корректности Фробениуса. \cite{DS2000}

\section{Импульсная управляемая система}
\label{sec:ids}

Возникает необходимость корректно определить решения системы
\eqref{system_s}. Опишем вначале множество импульсных управлений. Пусть $\mu$ ---
ограниченная борелевская мера на $T$, удовлетворяющая условию
\begin{equation*}
  \mu(E) \in \mathbb{R}_+^2 \ \forall E \in \mathcal{B}_T
\end{equation*}
где $\mathbb{R}_+^2 = \{v \in \mathbb{R}^2 | v \ge 0\}$, $B_T$ ---
множество всех подмножеств отрезка $T$. Пусть $\mu_c$ --- непрерывная
состовляющая в разложении Лебега меры $\mu$,
$S_d(\mu) = \{ s \in T | \mu(\{s\}) \neq 0\}$ --- множество
импульсов. Сопоставим мере $\mu$ набор измеримых функций
$\gamma(\mu) = \{ \omega_s(\cdot) \}_{s\in S_d(\mu)}$, компоненты
которого $\omega_s(\cdot)$ являющиеся парами
$\big(\omega_{1s}(\cdot),\omega_{2s}(\cdot)\big)$ определены на
соответсвующих отрезках $[0,d_s]$, где $d_s := \mu(\{s\})$, и
удовлетворяют условиям:
\begin{itemize}
  \item $\omega_{1s}(\tau) \ge 0$, $\omega_{2s}(\tau) \ge 0$, 
    $\omega_{1s}(\tau) + \omega_{2s}(\tau) = 1$, $\tau \in [0,d_s]$,
  \item $\displaystyle\int_{0}^{d_s} \omega(\tau) d\tau = \mu(\{s\})$.
\end{itemize}
Пару $(\mu,\gamma(\mu))$ будем называть импульсным управлением и
обозначать символом $\pi(\mu)$.

Введем понятие ресурса импульсного управления. Для векторной меры
$\mu$ полную вариацию обозначим через $|\mu|$ и зададим правилом
\begin{equation*}
  |\mu|(E) = |\mu_1|(E) + |\mu_2|(E) \ \forall E \in \mathcal{B}_T,
\end{equation*}
где $\mu_1,\mu_2$ --- компоненты $\mu$, причем в силу
неотрицательности $|\mu_i| = \mu_i$, $i=1,2$. Тогда ресурсом
управления $\pi(\mu)$ назовем величину $|\mu|(T)$, а ограничением на
ресурс импульсного управления --- неравенство

\begin{equation}
  \label{eq:mu_res}
  |\mu|(T) \le V,
\end{equation}
где $V \ge 0$ --- заданное действительное число. Множество импульсных
управлений удовлетворяющих ограничению \eqref{eq:mu_res}, будем
обозначать $\mathcal{W}(T)$.


Теперь, мы накопили достаточно материала для описания импульсной
системы. Рассмотрим систему \eqref{system_d} соответствующую системе
\eqref{system_s}.

\begin{equation}
  \label{system_d}
  \begin{array}{l}
    dx(t)=f\big(t,x(t)\big)dt+G\big(t,x(t)\big)\pi(\mu), \quad
    x(0)=x_0, \\[8pt]
    \pi(\mu) \in \mathcal{W}(T)
  \end{array} \tag{$\mathcal{D}$}
\end{equation}

Здесь $T=[0,t_1]$ --- заданный отрезок времени, $x(\cdot)$ ---
непрерывная справа на $(0,t_1]$ функция ограниченной вариации, $x(t)
\in \mathbb{R}^2$. Решения системы \eqref{system_d}, соответсвующие
управлению $\pi(\mu)$, понимаются как решения интегрального уравнения
с мерой
\begin{equation*}
  x(t) = x_0 + \int_0^t f(t,x(t))dt + \int_0^t G(t,x(t)) \mu_c(dt) +
  \sum_{s \le t,\\s \in S_d(\mu)} (z_s(d_s) - x(s-)), \quad t \in (0,t_1]
\end{equation*}
где для каждого $s \in S_d(\mu)$ функция $z_s(\cdot)$ --- решение
дифференциального уравнения
\begin{equation*}
  \frac{dz_s(\tau)}{d\tau} = G(s,z_s(\tau))\omega_s(\tau), \quad z_s(0)=x(s-)
\end{equation*}

Таким образом, при расширении системы \eqref{system_s} и переходе к
соответствующей импульсной системе \eqref{system_d} мы к обычным
(абсолютно непрерывным) траекториям добавляем все частичные
поточечные пределы последовательностей обычных
траекторий. Полученное множество будем называть множеством
траекторий импульсной системы, соответствующей \eqref{system_s} и будем
обозначать импульсную систему символом \eqref{system_d} Поясним,
что если в системе \eqref{system_d} траектории могут иметь скачки,
то их обобщенные производные будут содержать дельтообразные
составляющие --- импульсы; следовательно, импульсы появятся в
соответствующих управлениях, отсюда и названия импульсной системы и
импульсного управления.

Приведем определение траекторий импульсной системы \eqref{system_d}

{\bf Определение 1.}
  {\it Функция ограниченной вариации $x(\cdot)$ называется траекторией
  импульсной системы \eqref{system_d} (или обобщенным (разрывным)
  решением системы \eqref{system_s}), если найдется такая последовательность
  траекторий системы \eqref{system_s} $\bigl\{x_k(\cdot)\bigr\}$, что выполняется
  условие 
\begin{equation*}
  x_k(t)\to x(t) \quad  \forall \, t\in [0,T].
\end{equation*}}

\section{Множество достижимости импульсной управляемой системы}
\label{sec:rsids}


Определим \emph{множество достижимости} в момент  времени $t=t_1$
импульсной системы \eqref{system_d}. Обозначим это множество через 
$ {\mathcal R}_V(t_1)$. Это множество из  пространства ${\mathbb R}^n$
(конечномерное  множество), оно состоит из точек $x_b,$ в  которые в
момент $t=t_1$ попадают траектории  системы \eqref{system_d}, выходящие
в начальный  момент из начальной точки $x_0$; точнее
\begin{equation*} 
  {\mathcal R}_V(t_1):=\left\{ x(t_1+) \ \big| \
    x(\cdot) \mbox{ -- траектория системы } ({\mathcal D}) \mbox{ в
      смысле опр. 1} \right\}.
\end{equation*}

Укажем важные свойства множества достижимости \cite{ZS1991, SS2010}:

\begin{itemize}
\item[ 1)] Множество достижимости ${\mathcal R}_V(t_1)$ является
компактным множеством в ${\mathbb R}^n$.
\item[ 2)] Множество достижимости ${\mathcal R}_V(t_1)$ является связным
множеством в ${\mathbb R}^n,$ т.е. его нельзя представить в виде
объединения двух непустых непересекающихся подмножеств.
\item[ 3)] Множество достижимости ${\mathcal R}_V(t_1)$ непрерывно
зависит от величины интегрального ресурса управления $V$. Напомним,
что множество ${\mathcal R}_m(t_1)$ непрерывно зависит от параметра $m$
при $m=V,$ если
\begin{equation*} 
  \lim_{m\to V} \rho\Big( {\mathcal R}_m(t_1) , {\mathcal
    R}_V(t_1) \Big) =0 ,
\end{equation*} где $\rho(A,B):=\max\bigl\{ \max\limits_{y\in
A}\min\limits_{z\in B} |y-z| ; \max\limits_{z\in B}\min\limits_{y\in
A} |y-z| \bigr\}$ --- хаусдорфово расстояние между двумя компактными
множествами $A$ и $B$. 
\end{itemize}

Эти свойства будут очень важны при численном построении внутренних
аппроксимаций множества достижимости.  В частности, свойство 3)
позволяет приближенно рассматривать множество достижимости как
некоторое конечное объединение множеств достижимости вспомогательной
системы, описанной в следующем разделе.

%%% Local Variables:
%%% mode: latex
%%% TeX-master: "rs-ids"
%%% End:
