\section*{Заключение}
\label{sec:conclusion}

В дипломной работе были рассмотрены два алгоритма аппроксимации
множеств достижимости для импульсных управляемых систем с билинейной
структурой. Алгоритмы реализованы в среде Scientific Python; выполнен
ряд численных экспериментов; проведено сравнение с аналитическими
оценками множества достижимости в тех случаях, когда это возможно;
найдена погрешность алгоритма (расстояние Хаусдорфа между множеством
достижимости и его аппроксимацией). На основе полученных данных
сделаны следующие выводы:

1) Переборный алгоритм может быть применен лишь для грубой
аппроксимации в наиболее простых случаях (размерность 1-2), поскольку
объем вычислений экспоненциально зависит от числа участков, на которые
разбит временной интервал.

2) Даже в относительно простых примерах время работы пиксельного
алгоритма на 1-2 порядка меньше, чем у переборного.

3) Пиксельный метод для систем с билинейной структурой весьма
эффективен: погрешность алгоритма во всех рассмотренных случаях близка
к шагу разбиения фазового пространства.

Следует отметить, что в рамках дипломной работы данные алгоритмы
применялись к довольно узкому классу задач. Некоторые расширения этого
класса (отсутствие ограничения на знак управлений) не требуют
существенной переработки алгоритмов и программ. Изменение нормы, в
которой задано ограничение на векторное управление, приведет к
заметному усложнению алгоритма.

Требует дальнейшего исследования вопрос об эффективном выборе
дискретизации фазового пространства: целесообразно измельчать сетку в
окрестности множества, на котором функция цены не дифференцируема.

%%% Local Variables:
%%% mode: latex
%%% TeX-master: "rs-ids"
%%% End:
