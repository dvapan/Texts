\section*{Заключение}
\addcontentsline{toc}{section}{Заключение}
\label{sec:conclusion}

В работе были представлены реализации численных методов решения
уравнения эйконала. Наиболее важным результатом является вариант
алгоритма FMM, предназначенный для аппроксимации множеств достижимости
импульсных управляемых систем. Рассмотрен ряд примеров, включая
вычисление расстояний на плоскости при неоднородной метрике и
восстановление формы изображения по черно-белому снимку. Разработаны
реализации на языке C. Проведены численные эксперименты с параллельным
вариантом FSM.

В задаче аппроксимации множеств достижимости разработанные программы
показывают себя с лучшей стороны, помогают в поисках аналитического
решения и позволяют создавать иллюстрации. В задаче восстановления
формы тела для синтезированных изображений задача решается с высокой
точностью; к сожалению, не удалось протестировать алгоритм в реальных
условиях. С учетом небольшого времени работы для актуальных постановок
задач использовать параллельную реализацию, как правило,
нецелесообразно; это потребуется, например, для аппроксимации множеств
достижимости систем в размерности 3 (в задаче 1) или при обработке
снимков высокого разрежения (в задаче 2). Изучение масштабируемости
параллельных реализаций представляет собой одно из возможных
направлений дальнейшей работы.

%%% Local Variables:
%%% mode: latex
%%% TeX-master: "eikonal_solver"
%%% End:
